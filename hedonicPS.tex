% use option [draft] for initial mission
%            [final] for the prepublication
\documentclass[ecta,nameyear,draft]{econsocart}
%
%\usepackage{}
\RequirePackage[colorlinks,citecolor=blue,linkcolor=blue,urlcolor=blue,pagebackref]{hyperref}

\startlocaldefs

%%%%%%%%%%%%%%%%%%%%%%%%%%%%%%%%%%%%%%%%%%%%%%
%%                                          %%
%% Uncomment next line to change            %%
%% the type of equation numbering           %%
%%                                          %%
%%%%%%%%%%%%%%%%%%%%%%%%%%%%%%%%%%%%%%%%%%%%%%
%\numberwithin{equation}{section}
%%%%%%%%%%%%%%%%%%%%%%%%%%%%%%%%%%%%%%%%%%%%%%
%%                                          %%
%% For Assumption, Axiom, Claim, Corollary, %%
%% Lemma, Theorem, Proposition, Hypothezis, %%
%% Fact                                     %%
%% use \theoremstyle{plain}                 %%
%%                                          %%
%%%%%%%%%%%%%%%%%%%%%%%%%%%%%%%%%%%%%%%%%%%%%%
\theoremstyle{plain}
\newtheorem{axiom}{Axiom}
\newtheorem{claim}[axiom]{Claim}
\newtheorem{theorem}{Theorem}[section]
\newtheorem{lemma}[theorem]{Lemma}
\newtheorem*{fact}{Fact}
%%%%%%%%%%%%%%%%%%%%%%%%%%%%%%%%%%%%%%%%%%%%%%
%%                                          %%
%% For Definition, Example, ,         %%
%% Notation, Property                       %%
%% use \theoremstyle{}                %%
%%                                          %%
%%%%%%%%%%%%%%%%%%%%%%%%%%%%%%%%%%%%%%%%%%%%%%
\theoremstyle{remark}
\newtheorem{definition}[theorem]{Definition}
\newtheorem*{example}{Example}

%%%%%%%%%%%%%%%%%%%%%%%%%%%%%%%%%%%%%%%%%%%%%%
%% Please put your definitions here:        %%
%%%%%%%%%%%%%%%%%%%%%%%%%%%%%%%%%%%%%%%%%%%%%%


\endlocaldefs

\begin{document}

\begin{frontmatter}

\title{Pecuniary Externalities and Hedonic Assets}
\runtitle{Pecuniary Externalities and Hedonic Asset Markets}

\begin{aug}
% use \particle for den|der|de|van|von (only lc!)
% [id=?,addressref=?,corref]{\fnms{}~\snm{}\ead[label=e?]{}\thanksref{}}
%
%% e-mail is mandatory for each author
%
%%% initials in fnms (if any) with spaces
%
\author[id=au1,addressref={add1}]{\fnms{Douglas}~\snm{Patton}\ead[label=e1]{douglaspatton@gmail.com}}
%\author[id=au2,addressref={add2}]{\fnms{Second}~\snm{Author}\ead[label=e2]{second@somewhere.com}}
%\author[id=au3,addressref={add2}]{\fnms{Third}~\snm{Author}\ead[label=e3]{third@somewhere.com}}
%%%%%%%%%%%%%%%%%%%%%%%%%%%%%%%%%%%%%%%%%%%%%%
%% Addresses                                %%
%%%%%%%%%%%%%%%%%%%%%%%%%%%%%%%%%%%%%%%%%%%%%%
\address[id=add1]{%
\orgdiv{ORISE Fellow participating at US EPA},
\orgname{Office of Research and Development}}


\end{aug}

%% Put support info here.  Reminder: do not thank the handling coeditor anonymously or by name
\support{We thank four anonymous referees. The first author gratefully acknowledges
financial support from the National Science Foundation through Grant XXX-0000000.}
%
\begin{abstract}
To address a lack of attention to the welfare effects of house price returns generated by public policies that enhance environmental quality, I develop a dynamic model for a hedonic housing asset traded in a competitive market. The model distinguishes between the firm that owns each house and the consumers who rent. I derive present-value welfare change measures for non-marginal and marginal changes in amenities that are consistent with the static measures currently available. The dynamic welfare measures provide additional details about the distribution of benefits over time that reveal the inherent pecuniary externalities in hedonic asset markets that result in a transfer of economic benefits of amenity improvements from renter to owner. I also develop and consider a model extension specific to owner-occupied housing, and briefly discuss the policy implications of these results in the contexts of benefit cost analysis and environmental justice.
\end{abstract}

\begin{keyword}
\kwd{First keyword}
\kwd{second keyword}
\end{keyword}

\end{frontmatter}
%%%%%%%%%%%%%%%%%%%%%%%%%%%%%%%%%%%%%%%%%%%%%%%%%%%%%%%%%%%%%%%%%%%%%%%%%
%%%% Main text entry area:
%%%%%%%%%%%%%%%%%%%%%%%%%%%%%%%%%%%%%%%%%%%%%%%%%%%%%%%%%%%%%%%%%%%%%%%%%

\section{Introduction}

\caps{Economic assessments of} ecosystem service values frequently focus on quantifying direct economic benefits to households or consumers. The consumer oriented approach often follows from the way ecosystem services directly affect human well-being. However, in addition to direct impacts on human well-being, there are often price changes that accompany changes in ecosystem service flows. This is particularly evident with the hedonic price method, which uses changes in market prices to assess the benefits associated with improvements in environmental amenities that benefit households. 

The hedonic price method was grounded in the context of partial equilibrium of competitive markets by Rosen (1974). For much of the history of the method, identification of economic benefits required estimating a second stage demand function or making additional assumptions about the stability of the hedonic function across non-marginal variations in amenities (Freeman et al. 2014). 

The relative price changes that accompany economic activity have been described as pecuniary external economies (Scitovsky 1954, Buchanan and Stubblebine 1962). Pecuniary externalities that arise due to market activity are not considered economically inefficient and do not contribute to deadweight loss. However, pecuniary externalities do affect well-being and when they arise as a consequence of public decision making, it is not appropriate to ignore the costs and benefits that spillover into related markets . 

The hedonic price method of assessing the private benefits of ecosystem services to the occupants of housing is an apt example of a methodology that typically omits pecuniary effects from consideration. As discussed below, the hedonic price method relies on indirect changes in housing prices to quantify the economic values that occupants place on environmental amenities. The fact that the unit of housing as a productive asset has increased in value due to the improved amenities is generally ignored in the environmental economics literature. 

In contrast to much of the environmental economics literature, other fields that consider changes in housing prices focus on price changes as they relate to the value of housing as an asset.  

and those price changes can be used to . These price changes are pecuniary externalities and part of the way efficient markets function and benefits or costs that come about as a consequence of them are known as pecuniary externalities. 



The consumer oriented approach is sensible because individual well-being is the fundamental concern of economics. Economic models allow for evidence of changes in consumption to be used to measure changes in well-being. and information about consumption opportunities obtained through  active and passive use of ecosystem services. Many questions about ecosystem services focus on services that are public goods (non-excludable and non-rival) or club goods (non-excludable and rival). This makes sense in the context of public decision making and regulation because even well-functioning private markets are not expected to provide efficient levels of public and club goods.  

However, understanding how people benefit from ecosystem services through the consumption of private goods (excludable and rival)  also receives attention. Access to private real estate is a prime example of a private good that may be necessary for enjoying certain excludable ecosystem services. Hedonic valuation studies are generally recognized as a means for quantifying the economic benefits of these excludable ecosystem service values. 

People can benefit from housing in two distinct ways: as the owner and as the renter. This distinction is embodied in the treatment of housing in national income accounting.  National Income Accounts include new houses in the Private Fixed Investment category along with new factories, in the Rental Income of Persons category as well as in the Personal Consumption Expenditure category with food and recreation https://www.bea.gov/resources/methodologies/nipa-handbook. These categorized measures of aggregate trade flows from housing provide a framework for using the hedonic price method for quantifying economic benefits related to housing. 

\section{Literature Review}

\subsection{Benefits of Amenities}

Freeman and more recently Freeman et al. are commonly cited as a source for theoretical underpinnings of the data and models used for estimating the benefits to housing consumers.  

Freeman (1979) describes the process of aggregating the benefits to households of marginal improvements in amenities as a process of summing welfare measures based on analysis of household demand. Freeman (1999) asks, “how can we use the information on prices and preferences that can be extracted from the hedonic housing market to calculate  measures of welfare change for changes in environmental amenities?” 



Smith and Huang (1995) conducted an early meta-analysis of hedonic studies relating to air quality and considered only MWTP as a source of benefits. They also define MWTP as the “change in the asset value of the house”. 

Zabel and Kiel (2000) discuss measuring the benefits to homeowners and their households” and focus exclusively on quantifying benefits related to consumer demand, as evidenced by their discussion of the indirect utility function when aggregating economic benefits for the purpose of benefit cost anaylsis. In their concluding remarks, Zabel and Kiel (2000) compare their results to Smith and Huang’s (1995) results while describing them as benefits to the home owners. 

Bui and Mayer (2003) quantify environmental amenity benefits by assessing their capitalization into housing prices. 

\cite{niskanen77} and \cite{freeman80} develop and extend dynamic hedonic models and extend them to quantify benefits in the presence of taxes. Around the same time, \cite{sonstelie80} develop a dynamic approach to hedonic modeling that incorporates interest rates and tax rates and the "user costs" a landlord experiences. \cite{hendershott83} distinguish between owners and renters when modeling the impacts of tax treatment on households. \cite{poterba84} take a similar approach and use the concept of user cost to assess how inflation and tax rates impact housing prices and investment in new construction in the context of macroeconomic asset markets, focusing their model on owner-occupied housing. More recently, Bishop and Murphy (2011, 2019) estimate dynamic models that include a home-buyer's expectations about future prices. None of these works distinguish between the well-being of the renter and the owner of a housing unit, though \cite{hendershott83} does distinguish between the income of a renter and an owner.


\subsection{External Economies} 

In Economics, pecuniary and technological externalities have been defined to characterize interactions among market as well as political participants (Holcombe and Sobel 2001). Scitovsky (1954) discussed external economies in the context of industrial organization, distinguishing between technological and pecuniary externalities. Buchanan and Stubblebine (1962) generalized technological externalities to include consumers. Technological externalities have a directly impact on an external party other than through prices while the many effects of price changes are pecuniary externalities. Holcombe and Sobel (2001) suggest that because pecuniary externalities do not result in economic inefficiency the associated losses and gains have received much less attention. The direct mentions of pecuniary externalities in recent literature (e.g., Dávila and Korinek 2018) are largely confined to macroeconomic discussions related to investment and finance. 

Externalities are also referred to as spillovers. Kelly (2011) discuss strategic spillovers where externalities are generated deliberately as a form of “extortion”. 

Investigations of pecuniary externalities in environmental economics often focus on indirect effects. The debate over the indirect effects of biofuel subsidies provides a template for considering some of the ways relative price changes may be included in policy design. Zilberman et al. (2011) examine the impact of biofuel subsidies on indirect land use change through increased food prices, and they begin by noting the validity of considering pecuniary externalities in policy design. In Zilberman et al (2011) and other studies such as Zilberman et al. (2013) a similar group of authors cast doubt on the inclusion of indirect land use change in biofuel subsidy planning largely because of the myriad other indirect effects that add to uncertainty and which should also be included in an ideal world but are left out in practice. 





\subsection{Environmental Justice} 

Improved environmental quality can lead to gentrification, displacing lower income households that tend to rent rather than own their homes (Banzhaf et al 2019). 	 





\section{Methods}
\subsection{Owners and Renters}
In this section, we develop a dynamic model of a profit maximizing firm that can buy and sell a unit of housing in an asset market and a utility maximizing household who can rent a unit of housing in a rental market. The unit of housing is a bundle of attributes, $z$, and a separate attribute, $q$, of primary interest which we might imagine is an environmental amenity such as clean air. We assume that the asset and rental markets are each in a competitive equilibrium and that participants have complete information, aside from any unexpected shocks.
We assume that the household is maximizing a utility function constrained by their budget and conditional on prices and income and that there is a corresponding indirect utility function. The households optimization problem is over a set of houses with continuously variable attributes, leading to the outcome where each household's marginal willingness to pay for an attribute is equal to the implicit rental price of that housing attribute, as will be discussed below.
The firm and the household make decisions at each time period to buy/sell and to rent. We focus our analysis on a typical housing unit which may be owned and rented by different firms and households each year. 
The household and the firm interact at the beginning of each time period when the household pays rent. The transaction is the result of the household maximizing the present value of their utility. Included in the household's first order conditions would be the  
The firm purchases a house at time $t$ for $P_t$ borrowed at interest rate $r$. Later, at $t+1$, the firm sells the house for $P_{t+1}$ and repays the loan plus interest, $(1 + r)P_t$. After purchasing the house, the firm rents
out the unit of housing to a household for, $R_t$ and pays costs $C_t$. The present value of the producer’s profit for a single time period (assuming no fixed costs) will be,
\begin{equation}
	\pi_t = (R_t-C_t)+\left(\frac{P_{t+1}}{1+r}-\frac{P_t(1+r)}{1+r}\right).\label{pi1}
\end{equation}
Where the first term is the net revenue from rent and costs for the time period and the second term
represents the present value of house price returns and the principle plus the interest payment on $P_t$.

Over the same time period, a household rents the house for $R_t$ and has a willingness to pay for housing, $\mathit{WTP}_t=\mathit{WTP}(q_t,z_t,y)$ for that same housing unit. For simplicity, we assume that consumers have identical WTP functions conditional on income and housing attributes. We can assume that there is an equilibrium rent function, $R_t=R(q_t,z_t,P_t,C_t)$ and that $\mathit{WTP}_t \geq R_t$. One might estimate this function using the hedonic price method and time series rent data, though as we will show this is not necessary and house prices can be used to quantify the welfare impacts of a policy or treatment. 
At the end of the time period, the consumer’s monetized benefit is:
\begin{equation*}
	 w_t=\mathit{WTP}_t-R_t
\end{equation*}
Here we can see that optimization will result in a first order condition that reveals marginal willingness
to pay for a single point on the household’s demand curve for housing in time period, $t$,
\begin{equation}
	\frac{\partial \mathit{WTP}_t}{\partial q_t}=\frac{\partial R_t}{\partial q_t} \label{foc}
\end{equation}
Returning to the firm, if we assume the market uses information to price houses efficiently, today’s price will equals tomorrow’s
price,
\begin{equation*}
	P_t=P_{t+1}.
\end{equation*}
Future work might consider relaxing this assumption by incorporating the impacts of a stochastic
specification and including short run macroeconomic effects.
We can rewrite the profit equation in \ref{pi1} as,
\begin{equation}
	\pi_t=(R_t-C_t)-P_t*\frac{r}{1+r}.\label{pi1.1}
\end{equation}
Further, if we assume $\pi_t=0$ and rearrange to solve for the equilibrium rent,
\begin{equation}
	R_t=C_t+P_t*\frac{r}{1+r}.\label{rent1}
\end{equation}
Which says that over a time period rent is equal to costs during the time period plus the present value of
the interest payment on the cost of the house as capital. If rent were lower than this value, profits
would be negative, firms would exit, house prices would fall, and the interest payments and other costs
would decline until profits were no longer negative.
For each housing unit we can characterize the total welfare, $W_t$ associated with owning and renting the house for a time period,
\begin{eqnarray*}
	W_t & = & w_t+\pi_t,\\
	& = & \mathit{WTP}_t-R_t+0.
\end{eqnarray*}
\subsection{Assessing a Shock}
We define a treatment that happens at the end of $t_0$ that causes $q$ to increase uniformly such that $\Delta q_t=q'_t-q_t>0$ for $t>0$.
For consumers the alternate welfare measure based on willingness to pay for the higher amenity level,$\mathit{WTP}(q'_t,z_t,y)=\mathit{WTP'}_t$, and an adjusted equilibrium rent, $R'_t=R(q'_t,z_t,P'_t,C'_t)$ is,
\begin{equation*}
	w'_t=\mathit{WTP'}_t-R'_t \text{ for } t>0
\end{equation*}
The difference between the treated and untreated states for the consumer is,
\begin{eqnarray}
	\Delta w_t&=&w'_t-w_t=WTP'_t-WTP_t-(R'_t-R_t)\\
	&=&\Delta\mathit{WTP}_t-\Delta R_t
\end{eqnarray}
For the producer’s alternative profit in the treated state of reality we write,
\begin{equation*}
	\pi'_t = (R'_t-C'_t)+\left(\frac{P'_{t+1}}{1+r}-\frac{P'_t(1+r)}{1+r}\right)\text{ for } t>0.
\end{equation*}
which can be simplified in the manner of \ref{pi1.1} by assuming competition returns profits to zero and markets function efficiently,
\begin{equation}
\pi'_t = (R'_t-C'_t)-P'_t*\frac{r}{1+r}\text{ for } t>0.\label{pi2}
\end{equation}

The difference between the treated and untreated states for the producer is zero in all subsequent periods due to the effects of competition, but in the period where the treatment occurs,
\begin{equation}
	\Delta\pi_0=\frac{P'_1-P_0}{1+r}=\frac{\Delta P_1}{1+r}\label{pitzero}
\end{equation}
Now we can calculate the present value of the change in total welfare due to the treatment,
\begin{eqnarray}
	\Delta W_{\mathrm{PV}}&=&\Delta \pi_0+\sum_{t>0}\Delta w_t\nonumber\\
	&=&\frac{\Delta P_1}{1+r}+\sum_{t>0} (\Delta \mathit{WTP}_t-\Delta R_t)(1+r)^{-t} \label{deltaW1}\\
	&=&\Delta P_{\mathrm{PV}}+\Delta\mathit{WTP}_\mathrm{PV}-\Delta R_{\mathrm{PV}}\label{deltaW2}
\end{eqnarray}
This is an intriguing result that helps refine how we compute the distribution of benefits from public projects that enhance amenities. The house price return is an unexpected profit, enjoyed by the owner. This benefit is immediate and has a clear value, while the sign on the individual differences in the sum of future consumer benefits is indeterminate. This result may seem inconsistent with the usual measure of welfare change, such as in \cite{freeman14} and \cite{banzhaf20}, but next we consider the pattern of changing rent that accompanies the change in price to derive a more familiar measure. 

While $\Delta\pi_t=0$ for $t>0$, it is still helpful to consider the equation for the change in profits because it can help us understand how rents will change,
\begin{eqnarray}
	\Delta \pi_t &=& \Delta R_t-\Delta C_t-\Delta P_t\left(\frac{r}{1+r}\right)=0 \text{ for } t>0\nonumber\\
	\Delta R_t &=& \Delta C_t+\Delta P_t \left(\frac{r}{1+r}\right)  \text{ for } t>0\label{deltarent}
\end{eqnarray}
Returning to our measure of welfare change in equation \ref{deltaW1}, we can substitute for $\Delta R_t$  from \ref{deltarent} and rearrange,
\begin{eqnarray}
	\Delta W_{\mathrm{PV}}&=&\frac{\Delta P_1}{1+r}+\sum_{t>0} \left(\Delta \mathit{WTP}_t-\left(\Delta C_t+\Delta P_t \left(\frac{r}{1+r}\right)\right)\right)(1+r)^{-t} \nonumber\\
	%&=&\frac{\Delta P_1}{1+r}+\sum_{t>0} (\Delta \mathit{WTP}_t-\Delta C_t)(1+r)^{-t}-\Delta P_1 \sum_{t>0}  \left(\frac{r}{1+r}\right)(1+r)^{-t}\nonumber\\
	&=&\frac{\Delta P_1}{1+r}+\sum_{t>0} (\Delta \mathit{WTP}_t-\Delta C_t)(1+r)^{-t}-\frac{\Delta P_1}{1+r}\nonumber\\
	&=&\Delta \mathit{WTP}_{\mathrm{PV}}-\Delta C_{\mathrm{PV}} \label{deltaW2}
\end{eqnarray}
Which is the dynamic counterpart to the usual welfare measure. It is important to note that this measure does not convey the imbalanced flow of benefits to the owner and the renter. Indeed, because the public policy that improved the amenity was targeting consumers who would directly benefit, it is a pecuniary externality that competition among renters allows the homeowner to capture a large share of the benefits through rising rent that propels a rising asset value.

\subsection{Marginal Effects}
We are ultimately interested in understanding the relationship between price changes and welfare measures. In the usual approach that is static and does not distinguish between owner and renter, consumer optimization leads to a first order condition,$\frac{\partial{\mathit{WTP}}}{\partial q}=\frac{\partial P_h}{\partial q}$, which is the theoretical result that connects observable prices to WTP. I take a similar approach using the dynamic model described above, but with an infintesimally small treatment, $\partial q_{t>0}$ applied to all future time periods. I rely on the first order condition in equation \ref{foc} from the renter's optimal choice to reveal information about the consumer's marginal willingness to pay for $q$,

\begin{eqnarray}
\frac{\partial \mathit{WTP}_{\mathrm{PV}}}{\partial q_{t>0}}&=&\sum_{t>0} \frac{\partial \mathit{WTP}_t}{\partial q_t}(1+r)^{-t}\\ \label{pvmwtp}
&=&\sum_{t>0} \frac{\partial R_t}{\partial q_t}(1+r)^{-t}\nonumber\\
&=&\sum_{t>0} \frac{\partial C_t}{\partial q_t}(1+r)^{-t}+\frac{r}{1+r}\sum_{t>0} \frac{\partial P_t}{\partial q_t}(1+r)^{-t}\nonumber
\end{eqnarray}
Which can be rearranged to help us understand the partial derivative of the price function,
\begin{equation}
	\frac{\partial P_1} {\partial q_{t>0}}=\frac{\partial \mathit{WTP}_{\mathrm{PV}}}{\partial q_{t>0}}-\frac{\partial C_{\mathrm{PV}}}{\partial q_{t>0}} \label{marginalPrice}
\end{equation}
If we assume that costs do not change, then we have the standard tangency condition used in the static ownership literature.

\subsection{Renters Who Own}
If we consider renters who also own a home (though not necessarily the one they rent), we can consider the effects on the market of increased wealth due to a policy or treatment that increases profits. If we consider the willingness to pay for housing function and that increases in wealth translate to smooth increases in consumption in perpetuity (implying the household's marginal propensity to consume (MPC) equals one), then income becomes a function of attributes of the house, such as q. So we can define WTP for renters who own,
\begin{eqnarray*}	
	\mathit{WTP}_t^{\mathit{owner}}&=&\mathit{WTP}(q_t,z_t,y_t(q_t)) \text{ and} \\
	\mathit{WTP'}_t^{\mathit{owner}}&=&\mathit{WTP}(q'_t,z_t,y(q'_t)) \text{ for } t>0\\
	&=&	\mathit{WTP}(q'_t,z_t,y+\pi_t*r) \text{ for } t>0
\end{eqnarray*}
Notably, this implies that the previous WTP measures are implicitly for renters who do not own a house.



Now we can consider the marginal welfare measure for this market of renter-owners,

\begin{eqnarray}
	\frac{\partial \mathit{WTP}^{\mathit{owner}}_{\mathrm{PV}}}{\partial q_{t>0}}&=&\sum_{t>0} \frac{\partial \mathit{WTP}^{\mathit{owner}}_t}{\partial q_t}(1+r)^{-t} \nonumber\\
	&=&\sum_{t>0}\frac{\partial \mathit{WTP}_t^{\mathit{owner}}}{\partial q_t}|_{\partial y=0}*(1+r)^{-t} \nonumber\\
	&& +\sum_{t>0}\frac{\partial \mathit{WTP}_t^{\mathit{owner}}}{\partial y_t}*\frac{\partial y_t}{\partial q_t}(1+r)^{-t} \nonumber\\
	%&=&\sum_{t>0}\frac{\partial \mathit{WTP}_t^{\mathit{renter}}}{\partial q_t}*(1+r)^{-t}\nonumber\\
	%&& +\sum_{t>0}\frac{\partial \mathit{WTP}_t^{\mathit{renter}}}{\partial y_t}*\frac{\partial y_t}{\partial q_t}(1+r)^{-t} \nonumber
	&=&\sum_{t>0}\frac{\partial \mathit{WTP}_t}{\partial q_t}*(1+r)^{-t}\nonumber\\
	&& +\sum_{t>0}\frac{\partial \mathit{WTP}_t}{\partial y_t}*\frac{\partial y_t}{\partial q_t}(1+r)^{-t} \nonumber
\end{eqnarray}

Next we can assume $\partial \mathit{WTP}_t/\partial y_t=\lambda$, where $\lambda$ is constant over time and write,
\begin{eqnarray}
	\frac{\partial \mathit{WTP}^{\mathit{owner}}_{\mathrm{PV}}}{\partial q_{t>0}} 
	&=& \sum_{t>0}\frac{\partial \mathit{WTP}_t}{\partial q_t}|_{\partial y=0}*(1+r)^{-t} +
	\sum_{t>0}\frac{\partial \mathit{WTP}_t}{\partial y_t}*\frac{\partial y_t}{\partial q_t}(1+r)^{-t} \label{mwtpOwner0}\\
	&=& \sum_{t>0}\frac{\partial \mathit{WTP}_t}{\partial q_t}|_{\partial y=0}*(1+r)^{-t} +
	 \sum_{t>0}\frac{\partial \mathit{WTP}_t}{\partial y_t}*\frac{r}{1+r}*\frac{\partial P_t}{\partial q_t}(1+r)^{-t} \nonumber\\
	&=&\frac{\partial \mathit{WTP_{\mathrm{PV}}}}{\partial q_{t>0}} + \frac{\lambda}{1+r}*\frac{\partial P_1}{\partial q_{t>0}} \label{mwtpOwner1}
\end{eqnarray}
Where we have used $\frac{\partial y_t}{\partial q_t}=\frac{\partial \pi_t}{\partial q_t}$. Next, the first term in the right hand side of \ref{mwtpOwner1} describes the renter's experience in the rental marketplace (whether or not they are an owner), so we can solve equation \ref{marginalPrice} for marginal WTP and substitute it into the first term of \ref{mwtpOwner1},

\begin{eqnarray*}
	\frac{\partial \mathit{WTP}^{\mathit{owner}}_{\mathrm{PV}}}{\partial q_{t>0}} =	\frac{\partial C_{\mathrm{PV}}}{\partial q_{t>0}}+\frac{1+r+\lambda}{1+r}*\frac{\partial P_1}{\partial q_{t>0}}
\end{eqnarray*}
which can be rearranged to highlight the relationship between marginal willingness to pay and marginal sale price for permanent changes in an amenity,
\begin{equation*}
	\frac{\partial P_1} {\partial q_{t>0}}=\left(
	\frac{\partial \mathit{WTP}^{\mathit{owner}}_{\mathrm{PV}}}{\partial q_{t>0}}-\frac{\partial C_{\mathrm{PV}}}{\partial q_{t>0}}\right)\left(\frac{1+r}{1+r+\lambda}\right)
\end{equation*}


When the renter-owner is choosing the optimal location, there is no change in profits, so the tangency condition in equation \ref{foc} implied by a utility maximizing renter is not affected by ownership. However, an increase in $q$ such as a shock or treatment that affects the buyer's $WTP$ does affect the owner's profits. This leads to the situation where the marginal price under-estimates the benefit of the amenity change to renter-owners. Importantly, because an renter-owner's decision about where to rent does not influence the profits from homeownership, the wealth impacts on price that result from this are smaller than otherwise. Given our assumption of an MPC equal to one, pecuniary externalities do not transmit wealth effects into higher rents. To the extent that the MPC is below one and households do accumulate wealth, we expect wealth effects to drive up rent and house prices in a pattern analogous to multiplier effects in short-run macroeconomic analysis (\cite{samuelson39}), particularly in locations where renter-owners tend to live.

\subsection{Taxes}
It can be useful to separate taxes out as a component of costs to consider how changes in property tax rates impact renters and owners. Assume firms pay a property tax, $T_t$ at the end of each time period on $P_{t+1}$, where to $T_t=g*P_{t+1}$. We can incorporate this into the firm's profit equation from \ref{pi1},
\begin{equation}
	\pi^T_t = (R^T_t-C^T_t)+\left(\frac{P^T_{t+1}(1-g)}{1+r}-\frac{P^T_t(1+r)}{1+r}\right).\label{pi1T}
\end{equation}
The tax is imposed at the end of $t=0$. Markets subsequently adjust to a new equilibrium with competition returning profits to zero and allowing us to express equilibrium rents as a function of the new tax,
\begin{equation*}
	R^T_t=C^T_t-\left(\frac{P^T_{t+1}(1-g)}{1+r}-P^T_t\right) \text{ for }t>0
\end{equation*}
Then efficient use of information and competition cause prices in the next period to equal prices in this period, allowing us to simplify the rental equation,
\begin{equation*}
R^T_t=C^T_t-\frac{P^T_{t}(g+r)}{1+r}\text{ for }t>0.
\end{equation*}
Because the tax does not influence decisions made at the margin, we assume rents and costs do not change due to the property tax. 


The firm's profits are driven by competition to zero in subsequent years, but in year 0 the firm's profits will be non-zero due to the adjustment in price after $t=0$ that is a consequence of the tax,
\begin{equation}
	\pi^T_0 = (R^T_0-C^T_0)+\left(\frac{P^T_{1}(1-g)}{1+r}-P_0\right).\label{piT0}
\end{equation}
Because of our assumption that rents and costs do not change due to the property tax, we can solve for the change in price due to the tax,
\begin{eqnarray*}
	\Delta^T R_t&=&\Delta^T C_t+P^T_{t}\frac{g+r}{1+r}-P_{t} \frac{r}{1+r},\\
	0&=&0+P^T_{t}\frac{g+r}{1+r}-P_{t} \frac{r}{1+r},\\
	P^T_t&=& P_t\frac{r}{g+r}.
\end{eqnarray*}
Where $\Delta^T$ denotes the difference between the state of reality with the tax and without. As long as g and r are positive, the property tax will cause the price of the house to fall to a new equilibrium level.
Next we can consider the change in profits during $t=0$ as a function of $g$ and $P_0$,
\begin{eqnarray*}
	\Delta^T \pi_0 &=& P_1^T\frac{1-g}{1+r}-P_1\frac{1}{1+r}\\
	&=& P_1\left(\frac{r(1-g)}{(g+r)(1+r)}-\frac{1}{1+r}\right)\\
	&=&-P_0\frac{g}{g+r}
\end{eqnarray*}
Where we have relied on equality of prices between the two scenarios in $t=0$ and also that without a tax $P_t=P_0$. The total welfare impact on the firm and household is simply this change in profits in time period zero.
Given that rents do not change but the interest payment on the price of the house decrease, we can decompose the change in rents due to the tax,
\begin{eqnarray*}
	\Delta^T R_t&=&P^T_t\frac{g+r}{1+r}-P_t\frac{r}{1+r}\\
	&=&P^T_t\frac{g}{1+r}-(P_t-P_t^T)\frac{r}{1+r}.
\end{eqnarray*}
The first term is the present value of the annual tax payment and the second term is the rental payment reduction due to a lower house price. 
\subsection{A Corrective Tax}
While pecuniary externalities are not an inefficiency that merits correction on the grounds of economic efficiency, political and ethical perspectives may call for such a correction. For example, a public expenditure that enhances air quality is likely designed to benefit the households in a region, regardless of whether or not they own a house. We are interested in identifying a property tax rate that can be designed to exactly offset the pecuniary externality that transfers the benefits of the amenity change from renter to owner.

Consider an increase in amenities that causes a change in welfare, as described in \ref{deltaW2}. Next, let $T_t$ for $t>0$ be a tax on the home owner, and let, $S_t$ for $t>0$, be an equivalent subsidy paid to the renter. We can incorporate these into the welfare measure in \ref{deltaW2},
\begin{eqnarray*}
	\Delta W_{\mathrm{PV}}&=& \left(\Delta P_{\mathrm{PV}}-T_{\mathrm{PV}}\right)+\left(\Delta\mathit{WTP}_\mathrm{PV}-\Delta R_{\mathrm{PV}}+S_{\mathrm{PV}}\right).\nonumber\\
\end{eqnarray*}
To offset the owner's unexpected profits due to the increased amenities in \ref{pitzero}, we set the tax equivalent to the change in profits, that come about due to the change in amenities,
\begin{eqnarray*}
 \Delta \pi_{\mathrm{PV}}&=&T_{\mathrm{PV}}\\
\frac{\Delta P_1}{1+r}&=&P_1\frac{g}{g+r}
\end{eqnarray*}
which can be solved for the corrective property tax rate, $g$,
\begin{equation}
	g=\frac{\% \Delta P_1*r}{1+r-\% \Delta P_1} \label{gcorrect}
\end{equation}
Because hedonic models are often estimated with a log-transformed dependent variable, estimating \ref{gcorrect} is straightforward. To avoid the situation where the firm captures the benefits of the subsidy, the payment of the subsidy would need to be independent of the renter's decision about where to live. Renter-owners would experience no price change and they would receive a subsidy as well as pay a tax. Firms that own multiple houses would pay the tax multiple times while households receive all of the subsidies. Firms would experience no change in profits because the pecuniary impact of the corrective tax would cancel out the pecuniary impact of the amenity improvement.
Notably, this policy is revenue neutral for the government and the firm because, 
\begin{equation}
	P_0*\frac{g}{g+r}=T_{\mathrm{PV}}=S_{\mathrm{PV}}=\Delta \pi_{\mathrm{PV}}=-\Delta^T \pi_{\mathrm{PV}}=\sum_{t>0}\frac{g*P_t^T}{(1+r)^{-t}}.
\end{equation}
Where the first term is the change in profits from the tax and the last term is the government's tax revenue. 

 







\section{Discussion}
Throughout this paper we have mostly ignored the methodological challenges associated with quantifying changes in WTP that dominate the literature. It seems likely to us that the way future willingness to pay, future rents, future costs, etc are capitalized into today's prices increase the challenge associated with identifying an individual's demand. At the same time, we expect that the enhanced clarity that comes from separating the renter from the owner will ultimately determine the appropriate technique for quantifying Hicksian and Marshallian measures of welfare change.

We emphasize that the capitalized values related to housing, e.g., the price of the house, are the result of expected interactions between households and firms with probabilities of buying/selling and renting/moving that are generally strictly between zero and one. The non-zero probability of moving that renters face each time period is built into consumer decision making and is thus reflected by housing prices. In the context of horizontal sorting equilibrium models for occupants of housing, the expectations of firms and households about this process are built into observed prices. Our simple model omits this information, which we expect would lead to underestimation of benefits due to underestimated costs such as moving costs. It would be interesting to extend our model to include forward looking renters to better understand the impacts of frequent moving on renters \cite{bishop19} 

We find the idea of a corrective tax in the context of publicly funded enhancements to environmental amenities to be fascinating. While property taxes are familiar, subsidies targeting renters are uncommon, likely because they would effectively be an amenity that would be captured by the owner. A lump sum transfer to all resident in the affected region of the total tax receipt divided across all houses would be one approach to paying out the subsidy to renters. Renters in low-rent housing, who are likely to have a high marginal utility of income and a high marginal propensity to consume would also receive a larger subsidy than their landlords would pay in property taxes, a potential path toward restorative justice.

An important limitation of this work is the absence of investment, which we expect would be sensitive to property taxes, leading to a trade-off in well-being for residents of an area that institute a corrective tax as developed above.

While we have focused on single family homes in the development of our model, these results are applicable to any productive asset, such as a factory. For example, if an industry uses natural resources in production, then public enhancements to natural resources will tend to reduce production costs increasing profitability if competition . 

Extending the work of \cite{kanemoto88} to distinguish between renter-owners and renters in a dynamic, general equilibrium framework would be an interesting extension of this work.
\begin{thebibliography}{}
%


\bibitem[\protect\citeauthoryear{Banzhaf}{2020}]{banzhaf20}
\textsc{Banzhaf, S.} (2020):
``Panel Data Hedonics Rosen’s First Stage as a “Sufficient
Statistic",''
\textit{International Economic Review}, 61(2), 973--1000.
\endbibitem


\bibitem[\protect\citeauthoryear{Banzhaf, Ma, and Timmins}{2019}]{banzhafJustice19}
\textsc{Banzhaf, S., Ma, L. and Timmins, C.} (2019):
``Environmental Justice: The Economics of Race, Place, and Pollution,''
\textit{Journal of Economic Perspectives}, 33(1), 185--208.
\endbibitem

\bibitem[\protect\citeauthoryear{Bishop, et al.}{2020}]{bishop20}
\textsc{Bishop, K.C., Kuminoff, N.V., Banzhaf, H.S., Boyle, K.J., von Gravenitz, K., Pope, J.C., Smith, V.K. and Timmins, C.D., } (2020):
``Best practices for using hedonic property value models to measure willingness to pay for environmental quality,''
\textit{Review of Environmental Economics and Policy}, 14(2), 260--281.
\endbibitem

\bibitem[\protect\citeauthoryear{Bishop and Murphy}{2011}]{bishop11}
\textsc{Bishop, K.C. and Murphy, A.D.} (2011):
``Estimating the Willingness to Pay to Avoid Violent Crime: A Dynamic Approach,''
\textit{American Economic Review}, 101(3), 625--629.
\endbibitem

\bibitem[\protect\citeauthoryear{Bishop and Murphy}{2019}]{bishop19}
\textsc{Bishop, K.C., and Murphy, A.D.} (2019):
``Valuing time-varying attributes using the hedonic model: when is a dynamic approach necessary?,''
\textit{Review of Economics and Statistics}, 101(1), 134--145.
\endbibitem

\bibitem[\protect\citeauthoryear{Buchanan and Stubblebine}{1962}]{bs1962}
\textsc{Buchanan, J.M. and Stubblebine, W.C.} (1962):
``Externality,''
\textit{Economica}, 29(116), 371-384.
\endbibitem

\bibitem[\protect\citeauthoryear{Bui and Mayer}{2003}]{bui03}
\textsc{Bui, L.T. and Mayer, C.J.} (2003):
``Regulation and capitalization of environmental amenities: evidence from the toxic release inventory in Massachusetts,''
\textit{Review of Economics and statistics}, 85(3), 693--708.
\endbibitem

 
\bibitem[\protect\citeauthoryear{Dávila and Korinek}{2018}]{davila18}
\textsc{Dávila, E. and Korinek, A.} (2018):
``Pecuniary externalities in economies with financial frictions,''
\textit{The Review of Economic Studies}, 85(1), 352--395.
\endbibitem


\bibitem[\protect\citeauthoryear{Freeman}{1980}]{freeman80}
\textsc{Freeman, A.M.} (1980):
``Land Prices Substantially Underestimate the Value of Environmental Quality: A Comment,''
\textit{The Review of Economics and Statistics}, 62(1), 154--156.
\endbibitem


\bibitem[\protect\citeauthoryear{Freeman}{1999}]{freeman99}
\textsc{Freeman, A.M.} (1999):
\textit{The Measurement Of Environmental And Resource Values: Theory And Methods}.
Washington DC, U.S.A.: Resources for the Future.
\endbibitem
 
\bibitem[\protect\citeauthoryear{Freeman, Herriges, and Kling}{2014}]{freeman14}
\textsc{Freeman III, A.M., Herriges, J.A., and Kling, C.L.} (2014):
\textit{The Measurement Of Environmental And Resource Values: Theory And Methods}.
Washington DC, U.S.A.: Routledge
\endbibitem 

\bibitem[\protect\citeauthoryear{Hendershott and Slemrod}{1983}]{hendershott83}
\textsc{Hendershott, P.H. and Slemrod, J.} (1983):
``Taxes and the User Cost of Capital for Owner-Occupied Housing,''
\textit{Real Estate Economics}, 10(4), 375--393.
\endbibitem

, 2001. . , 29(4), pp.304-325. 
\bibitem[\protect\citeauthoryear{Holcombe and Sobel}{2001}]{holcombe01}
\textsc{Holcombe, R.G. and Sobel, R.S.} (2001):
``Public Policy Toward Pecuniary Externalities,''
\textit{Public Finance Review}, 29(4), 304--325.
\endbibitem


\bibitem[\protect\citeauthoryear{Kanemoto}{1988}]{kanemoto88}
\textsc{Kanemoto, Y.} (1988):
``Hedonic Prices and the Benefits of Public Projects,''
\textit{Econometrica}, 56(4), 981--989.
\endbibitem


, 1977. . , pp.375-377.
\bibitem[\protect\citeauthoryear{Niskanen and Hanke}{1977}]{niskanen77}
\textsc{Niskanen, W.A. and Hanke, S.H.} (1977):
``Land prices substantially underestimate the value of environmental quality,''
\textit{The Review of Economics and Statistics}, ??, 375--377.
\endbibitem


\bibitem[\protect\citeauthoryear{Palmquist}{1989}]{palmquist89}
\textsc{Palmquist, R.B.} (1989):
``Land as a Differentiated Factor of Production: A Hedonic Model and its Implications for Welfare Measurement,''
\textit{Land Economics}, 65(1), 23--28.
\endbibitem

\bibitem[\protect\citeauthoryear{Poterba}{1984}]{poterba84}
\textsc{Poterba, J. M.} (1984):
``Tax Subsidies to Owner-Occupied Housing: An Asset-Market Approach,''
\textit{The Quarterly Journal of Economics}, 99(4), 729--752.
\endbibitem


\bibitem[\protect\citeauthoryear{Samuelson}{1939}]{samuelson39}
\textsc{Samuelson, P.} (1939):
``Interactions between the Multiplier Analysis and the Principle of Acceleration,''
\textit{The Review of Economics and Statistics}, 62(2), 75--78.
\endbibitem

\bibitem[\protect\citeauthoryear{Scitovsky}{1954}]{scitovsky54}
\textsc{Scitovsky, T.} (1954):
``Two concepts of external economies,''
\textit{Journal of political Economy}, 21(2), 143--151.
\endbibitem

\bibitem[\protect\citeauthoryear{Sonstelie and Portney}{1980}]{sonstelie80}
\textsc{Sonstelie, J. C., and Portney, P. R.} (1980):
``Gross rents and market values: testing the implications of Tiebout's hypothesis,''
\textit{Journal of Urban Economics}, 7(1), 102-118.
\endbibitem


Zilberman, D., Barrows, G., Hochman, G. and Rajagopal, D., 2013. On the indirect effect of biofuel. American Journal of Agricultural Economics, 95(5), pp.1332-1337. 
\bibitem[\protect\citeauthoryear{Aumann}{1987}]{b13}
\textsc{Aumann, R. J.} (1987):
``Correlated Equilibrium as an Expression of Bayesian Rationality,''
\textit{Econometrica}, 55, 1--18.
\endbibitem
Zilberman, D., Hochman, G. and Rajagopal, D., 2011. On the inclusion of indirect land use in biofuel. U. Ill. L. Rev., p.413. 
\bibitem[\protect\citeauthoryear{Aumann}{1987}]{b14}
\textsc{Aumann, R. J.} (1987):
``Correlated Equilibrium as an Expression of Bayesian Rationality,''
\textit{Econometrica}, 55, 1--18.
\endbibitem

\bibitem[\protect\citeauthoryear{Aumann}{1987}]{b1}
\textsc{Aumann, R. J.} (1987):
``Correlated Equilibrium as an Expression of Bayesian Rationality,''
\textit{Econometrica}, 55, 1--18.
\endbibitem

\bibitem[\protect\citeauthoryear{Peck}{1994}]{b2}
\textsc{Peck, J.} (1994):
``Competition in Transactions Mechanisms: The Emergence of Competition,''
Unpublished Manuscript, Ohio State University.
\endbibitem

\bibitem[\protect\citeauthoryear{Enelow and Hinich}{1990}]{b3}
\textsc{Enelow, J., and M. Hinich}, eds. (1990):
\textit{Advances in the Spatial Theory of Voting}.
Cambridge, U.K.: Cambridge University Press.
\endbibitem

\bibitem[\protect\citeauthoryear{Wittman}{1990}]{b4}
\textsc{Wittman, D.} (1990):
``Spatial Strategies when Candidates Have Policy Preferences,''
in \textit{Advances in the Spatial Theory of Voting},
ed. by M. Hinich and J. Enelow.
Cambridge, U.K.: Cambridge University Press, 66--98.
\endbibitem

\bibitem[\protect\citeauthoryear{Cahuc, Postel-Vinay and Robin}{2006}]{b5}
\textsc{Cahuc, P., F. Postel-Vinay, and J.-M. Robin} (2006): 
``Supplement to `Wage Bargaining with On-the-Job Search: Theory and Evidence',''
\textit{Econometrica Supplementary Material}, 74.
\endbibitem
\end{thebibliography}

\end{document}
