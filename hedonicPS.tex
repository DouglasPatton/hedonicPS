% use option [draft] for initial mission
%            [final] for the prepublication
\documentclass[ecta,nameyear,draft]{econsocart}
%
%\usepackage{}
\RequirePackage[colorlinks,citecolor=blue,linkcolor=blue,urlcolor=blue,pagebackref]{hyperref}

\startlocaldefs

%%%%%%%%%%%%%%%%%%%%%%%%%%%%%%%%%%%%%%%%%%%%%%
%%                                          %%
%% Uncomment next line to change            %%
%% the type of equation numbering           %%
%%                                          %%
%%%%%%%%%%%%%%%%%%%%%%%%%%%%%%%%%%%%%%%%%%%%%%
%\numberwithin{equation}{section}
%%%%%%%%%%%%%%%%%%%%%%%%%%%%%%%%%%%%%%%%%%%%%%
%%                                          %%
%% For Assumption, Axiom, Claim, Corollary, %%
%% Lemma, Theorem, Proposition, Hypothezis, %%
%% Fact                                     %%
%% use \theoremstyle{plain}                 %%
%%                                          %%
%%%%%%%%%%%%%%%%%%%%%%%%%%%%%%%%%%%%%%%%%%%%%%
\theoremstyle{plain}
\newtheorem{axiom}{Axiom}
\newtheorem{claim}[axiom]{Claim}
\newtheorem{theorem}{Theorem}[section]
\newtheorem{lemma}[theorem]{Lemma}
\newtheorem*{fact}{Fact}
%%%%%%%%%%%%%%%%%%%%%%%%%%%%%%%%%%%%%%%%%%%%%%
%%                                          %%
%% For Definition, Example, ,         %%
%% Notation, Property                       %%
%% use \theoremstyle{}                %%
%%                                          %%
%%%%%%%%%%%%%%%%%%%%%%%%%%%%%%%%%%%%%%%%%%%%%%
\theoremstyle{remark}
\newtheorem{definition}[theorem]{Definition}
\newtheorem*{example}{Example}

%%%%%%%%%%%%%%%%%%%%%%%%%%%%%%%%%%%%%%%%%%%%%%
%% Please put your definitions here:        %%
%%%%%%%%%%%%%%%%%%%%%%%%%%%%%%%%%%%%%%%%%%%%%%


\endlocaldefs

\begin{document}

\begin{frontmatter}

\title{Pecuniary Externalities and Hedonic Assets}
\runtitle{Pecuniary Externalities and Hedonic Assets}

\begin{aug}
% use \particle for den|der|de|van|von (only lc!)
% [id=?,addressref=?,corref]{\fnms{}~\snm{}\ead[label=e?]{}\thanksref{}}
%
%% e-mail is mandatory for each author
%
%%% initials in fnms (if any) with spaces
%
\author[id=au1,addressref={add1}]{\fnms{Douglas}~\snm{Patton}\ead[label=e1]{douglaspatton@gmail.com}}
%\author[id=au2,addressref={add2}]{\fnms{Second}~\snm{Author}\ead[label=e2]{second@somewhere.com}}
%\author[id=au3,addressref={add2}]{\fnms{Third}~\snm{Author}\ead[label=e3]{third@somewhere.com}}
%%%%%%%%%%%%%%%%%%%%%%%%%%%%%%%%%%%%%%%%%%%%%%
%% Addresses                                %%
%%%%%%%%%%%%%%%%%%%%%%%%%%%%%%%%%%%%%%%%%%%%%%
\address[id=add1]{%
\orgdiv{ORISE Fellow participating at US EPA},
\orgname{Office of Research and Development}}


\end{aug}

%% Put support info here.  Reminder: do not thank the handling coeditor anonymously or by name
\support{This research was supported in part by an appointment to the ORISE Fellowship Program at the U.S. EPA, Office of Research and Development, Athens, GA, administered by the Oak Ridge Institute for Science and Education through Interagency Agreement No. DW8992298301 between the U.S. Department of Energy and the U.S. Environmental Protection Agency. This article has been reviewed by the U.S. Environmental Protection Agency and approved for publication. Mention of trade names or commercial products does not constitute endorsement or recommendation for use by the U.S. Government. The views expressed in this article are those of the authors and do not necessarily reflect the views or policies of the U.S. EPA.  The authors would like to thank internal and journal peer reviewers for many useful comments on the manuscript. All remaining errors are our own.}
%
\begin{abstract}
To address a lack of attention to the welfare effects of house price returns generated by public policies that enhance environmental quality, We develop a dynamic, discrete-time model for a hedonic housing asset and an associated flow of housing services. We assume the housing asset and rental services are traded in closely linked, competitive asset and rental markets, respectively. The model distinguishes between firms that own the asset and consumers who rent. Assuming an immediate transition to a new steady-state, we consider the effects on the firm and household of an unexpected shock, and we derive present-value welfare change measures for non-marginal and marginal changes in amenities, finding results consistent with the static measures currently available. However, the dynamic model provides additional details about the distribution of benefits over time, highlighting the inherent pecuniary externalities connecting rental and hedonic asset markets. These pecuniary externalities are fundamental to asset values and they transfer much of the economic benefits of amenity improvements from households to landlords. We also develop and consider a model extension specific to owner-occupied housing. We then consider the effects of property taxes and finally we derive a corrective tax and subsidy to offset the increases in rent from the pecuniary externality. 
\end{abstract}

\begin{keyword}
\kwd{First keyword}
\kwd{second keyword}
\end{keyword}

\end{frontmatter}
%%%%%%%%%%%%%%%%%%%%%%%%%%%%%%%%%%%%%%%%%%%%%%%%%%%%%%%%%%%%%%%%%%%%%%%%%
%%%% Main text entry area:
%%%%%%%%%%%%%%%%%%%%%%%%%%%%%%%%%%%%%%%%%%%%%%%%%%%%%%%%%%%%%%%%%%%%%%%%%

\section{Introduction}

\caps{Economic assessments of} ecosystem service values frequently focus on quantifying the direct economic benefits to households or consumers of changes to ecosystems. In empirical studies quantifying these values, revealed preference techniques quantify ecosystem service benefits by using information about observed market prices paid by consumers for either a privately traded ecosystem service or a privately traded good consumed alongside public ecosystem services. Because of the connection between ecosystem services and private markets, welfare changes include both consumers and firms. Additionally, applications of these techniques to residential housing services utilize asset prices rather than sale prices, a conflation of distinct markets that conceals the distribution of benefits between home owners and home occupants. The consequence of this conflation is that the economic benefits of public projects are enjoyed to a much greater extent by the asset owners relative to the consumers than standard analyses indicate. \cite{kuminoffpope14} say in their opening paragraph, "Homebuyers implicitly purchase the right to consume a bundle of local public goods when they buy a house." In the context of our analysis we would rephrase this as, "Homebuyers implicitly purchase the right to \textit{rent out} a bundle of local public goods when they buy a house."

In this paper, we contribute to the literature on the hedonic price method by developing a dynamic model of consumers and firms that includes competitive equilibria in an asset market and a closely related rental market. Our modeling approach emphasizes the adjustment process that follows unexpected shocks affecting residential real estate such as unexpected improvements in environmental amenities. Importantly, we develop a theoretical model that uses equilibrium in the rental market to reveal consumer preferences for housing amenities, which is in contrast to the common static approach that uses the price of the housing asset. Nonetheless, we derive measures of total welfare change based on the price of the housing asset that are consistent with the static literature (e.g., \cite{freeman99} and \cite{freeman14}). In the existing literature, much attention as been paid to estimating households' willingness to pay for improvements in environmental amenities as owner-occupants, though this assumption is often implicit. The increased granularity of our approach highlights the distributional consequences of improving hedonic assets such as housing, particularly for the approximately one third of households in the United States who rent and do not own. 

The connectedness of the rental and asset markets demonstrated in our modeling approach can be thought of as examples of pecuniary externality. And economists arguably tend to de-emphasize these types of spillovers from their analysis (\cite{holcombe01}). As discussed below, the hedonic price method relies on changes in housing sale prices to quantify the economic values that occupants place on environmental amenities. The fact that the unit of housing as a productive asset has increased in value at (and because of) the expense of renters due to the improved amenities is generally ignored when assessing ecosystem service benefits. The dynamic model we develop allows us to clearly separate the impacts on firms from the impacts on households; this distinction adds clarity to concerns about environmental justice that arise when public policies enhance property values, potentially leading to increased inequality and environmental gentrification (\cite{banzhaf20}). 

We also develop extensions to our dynamic model for households that own houses and for property taxes. Our results agree with existing dynamic models of property taxes in the literature, such as \cite{freeman80} and \cite{poterba84}, but with an entirely different motivation: environmental justice. We apply our model of property taxes to identify a  property tax rate and corresponding subsidy to consumers that affords policy makers an option for increasing environmental quality without harming current residents by allowing firms to capture the bulk of these benefits. We also include a brief empirical application of the tax to demonstrate the magnitude of the welfare transfers resulting from some recent hedonic studies. Key to our approach is the assumption that people have some basic human rights to environmental quality, and consequently private markets should not be used to exclude non-paying households from enjoying what is fundamentally theirs.


\section{Literature Review}

\subsection{Benefits of Amenities}

\cite{rosen74} and \cite{freeman74} contemporaneously established a static theoretical relationship between a household's utility and a hedonic price function for residential housing. \cite{freeman99} and more recently \cite{freeman14} are common references that continue to use the same theoretical justification for evaluating welfare changes. These authors found that hedonic price regressions were useful for estimating economic benefits because they revealed the implicit marginal price of and a household's willingness to pay for each of the bundled services provided by a house. 

The theoretical models developed by \cite{rosen74} and \cite{freeman74} treat each housing purchase as "pure consumption" (\cite{rosen74}), which makes the approach difficult to apply to renters who do not own a house. In their recent paper detailing best practices for hedonic assessments of ecosystem service values, \cite{bishop20} suggest a preference for rental-rate data where the proportion of owner-occupants is low, but they otherwise do not discuss the issue. In empirical applications, one option is to restrict analysis to owner-occupants. This approach is often followed by authors who seek to identify the homeowner's compensated demand curve. Empirical examples include \cite{palmquist84}, \cite{zabelkiel00}, and \cite{chaygreenstone05}. \cite{smithhuang95} focus their widely cited meta-analysis of hedonic valuation studies of clean air on owner-occupants. \cite{kuminoffpope14} include controls for "block group level ... percent owner occupied", but do not address how the measure might affect MWTP.
 
A number of recent studies do not address the distinction between owners and renters when reconstructing consumer demand curves. \cite{bishoptimmins18} do not address the question of ownership and discuss only home buyers, without mention of the occupants. Similarly \cite{banzhaf20} makes little mention of any distinction between renters and owners when approximating demand functions. Alternatively, a number of studies only seek to estimate implicit prices as estimates of marginal willingness to pay (MWTP) and any distinction between owners and renters is ignored (e.g., \cite{bajari12}, \cite{bui03}, and \cite{kuminoffpope14}).

A variety of theoretical dynamic models have been developed for hedonic analysis of real estate prices. These models tend to include more realistic treatment of taxation and interest rate issues but they lack a theoretical connection to a dynamic counterpart to MWTP. Early on, \cite{niskanen77} and \cite{freeman80} develop and extend dynamic hedonic models to quantify benefits in the presence of taxes. Around the same time, \cite{sonstelie80} develop a dynamic approach to hedonic modeling that incorporates interest rates and tax rates and the user costs a landlord experiences. \cite{hendershott83} distinguish between owners and renters when modeling the impacts of tax treatment on households. \cite{poterba84} take a similar approach and use the concept of user cost to assess how inflation and tax rates impact housing prices and investment in new construction in the context of macroeconomic asset markets, limiting their model to owner-occupied housing. More recently, Bishop and Murphy (2011, 2019) estimate dynamic models that include a consumer as a home-buyer's expectations about future prices. None of these works distinguish between the welfare of the renter and the owner of a housing unit, though \cite{hendershott83} does distinguish between the income of a renter and an owner. 


Bui and Mayer (2003) quantify environmental amenity benefits by assessing their capitalization into housing prices. 




\subsection{External Economies} 

In Economics, pecuniary and technological externalities have been defined to characterize interactions among market as well as political participants (Holcombe and Sobel 2001). Scitovsky (1954) discussed external economies in the context of industrial organization, distinguishing between technological and pecuniary externalities. Buchanan and Stubblebine (1962) generalized technological externalities to include consumers. Technological externalities have a directly impact on an external party other than through prices while the many effects of price changes are pecuniary externalities. Holcombe and Sobel (2001) suggest that because pecuniary externalities do not result in economic inefficiency the associated losses and gains have received much less attention. The direct mentions of pecuniary externalities in recent literature (e.g., Dávila and Korinek 2018) are largely confined to macroeconomic discussions related to investment and finance. 

Investigations of pecuniary externalities in environmental economics often focus on indirect effects. The debate over the indirect effects of biofuel subsidies provides a template for considering some of the ways relative price changes may be included in policy design. Zilberman et al. (2011) examine the impact of biofuel subsidies on indirect land use change through increased food prices, and they begin by noting the validity of considering pecuniary externalities in policy design. In Zilberman et al (2011) and other studies such as Zilberman et al. (2013) a similar group of authors cast doubt on the inclusion of indirect land use change in biofuel subsidy planning largely because of the myriad other indirect effects also left out.S



\subsection{Environmental Justice} 
\cite{banzhafJustice19} provide an overview of environmental justice in a spatial context that focuses on the use of hedonic price modeling for revealing preferences and disentangling the mechanisms that expose lower income households to higher levels of pollution. The same authors also recognize that For example, improved environmental quality can lead to gentrification, displacing lower income households that tend to rent rather than own their homes (Banzhaf et al 2019). 
 





\section{Theoretical Model}
\subsection{Owners and Renters}
In this section, we develop a dynamic model of profit maximizing firms that buy and sell units of housing in an asset market and utility maximizing households who rent units of housing in a rental market. Each unit of housing is a bundle of attributes, $z$, and a separate attribute, $q$, of primary interest which we might imagine is an environmental amenity such as clean air. We assume that the asset and rental markets are each in a competitive equilibrium and that participants have complete information, aside from any unexpected shocks.

We assume that the household is maximizing a utility function constrained by their budget and conditional on prices and income, and we assume that there is a corresponding indirect utility function. The households optimization problem is over a set of houses with continuously variable attributes, leading to the outcome where each household's marginal willingness to pay for an attribute is equal to the implicit rental price of that housing attribute for each time period, as will be discussed below.

The firm and the household make decisions at each time period to buy/sell and to rent. We focus our analysis on a typical housing unit which may be owned and rented by different firms and households each year. 
The household and the firm interact at the beginning of each time period when the household pays rent. The transaction is the result of the household maximizing the present value of their utility. Included in the household's first order conditions would be the  
The firm purchases a house at time $t$ for $P_t$ borrowed at interest rate $r$. Later, at $t+1$, the firm sells the house for $P_{t+1}$ and repays the loan plus interest, $(1 + r)P_t$. After purchasing the house, the firm rents
out the unit of housing to a household for, $R_t$ and pays costs $C_t$. The present value of the firm’s profit for a single time period will be,
\begin{equation}
	\pi_t = (R_t-C_t)+\left(\frac{P_{t+1}}{1+r}-\frac{P_t(1+r)}{1+r}\right).\label{pi1}
\end{equation}
Where the first term is the net revenue from rent and costs for the time period and the second term
represents the present value (i.e., at the start of the time period, $t$) of house price returns.

Over the same time period, a household rents the house for $R_t$ and has a willingness to pay for housing, $\mathit{WTP}_t=\mathit{WTP}(q_t,z_t,y_t)$ for that same housing unit, where $y_t$ is the renter's income. For simplicity in our mathematical exposition, we assume that consumers have identical WTP functions conditional on income and housing attributes. We  assume that there is an equilibrium rent function, $R_t=R(q_t,z_t,P_t,C_t)$ for houses in the market. One might estimate this function using the hedonic price method and time series rent data rather than housing price data, though as we will show this is not necessary and house prices can be used to quantify the welfare impacts of a policy or treatment. 

Each time period, the consumer’s monetized benefit is:
\begin{equation*}
	 w_t=\mathit{WTP}_t-R_t
\end{equation*}
Here we can see that optimization will result in a first order condition that reveals marginal willingness
to pay for a single point on the household’s demand curve for housing in time period, $t$,
\begin{equation}
	\frac{\partial \mathit{WTP}_t}{\partial q_t}=\frac{\partial R_t}{\partial q_t} \label{foc}
\end{equation}
Returning to the firm, if we assume the market uses information to price houses efficiently, today’s price will equals tomorrow’s
price,
\begin{equation*}
	P_t=P_{t+1}.
\end{equation*}
Future work might consider relaxing this assumption by incorporating the impacts of a stochastic
specification and including short run macroeconomic effects.
We can rewrite the profit equation in \ref{pi1} as,
\begin{equation}
	\pi_t=(R_t-C_t)-P_t*\frac{r}{1+r}.\label{pi1.1}
\end{equation}
Further, we assume arbitrage in the asset market leads to a steady-state with $\pi_t=0$, allowing us to solve for the equilibrium rent,
\begin{equation*}
	R_t=C_t+P_t*\frac{r}{1+r}.
\end{equation*}
Which says that over a time period rent is equal to costs during the time period plus the present value of
the interest payment on the cost of the house as capital. If rent were lower than this value, profits
would be negative, firms would exit, house prices would fall, and the interest payments and other costs
would decline until profits were no longer negative.

For each housing unit we can characterize the total welfare, $W_t$ associated with a firm owning and a household renting a house for a time period,
\begin{eqnarray*}
	W_t & = & w_t+\pi_t,\\
	& = & \mathit{WTP}_t-R_t+0.
\end{eqnarray*}
\subsection{Assessing a Shock}
We define a shock, such as an unexpected treatment, that improves environmental quality from $q$ to $q^\prime$ that happens at the end of $t=0$ causing $q$ to increase uniformly in future time periods such that $\Delta q_t=q^\prime_t-q_t>0$ for $t>0$.
The post-shock willingness to pay for the higher amenity level,$\mathit{WTP}^\prime_t=\mathit{WTP}(q^\prime_t,z_t,y)$ along with the post-shock equilibrium rent, $R^\prime_t=R(q^\prime_t,z_t,P^\prime_t,C^\prime_t)$ allows us to write the post-shock welfare measure,
\begin{equation*}
	w^\prime_t=\mathit{WTP}^\prime_t-R^\prime_t \text{ for } t>0
\end{equation*}
The difference between the treated and untreated states for the consumer is,
\begin{eqnarray*}
	\Delta w_t&=&w^\prime_t-w_t=\mathit{WTP}^\prime_t-\mathit{WTP}_t-(R^\prime_t-R_t)\\
	&=&\Delta\mathit{WTP}_t-\Delta R_t
\end{eqnarray*}
For the producer’s alternative profit in the treated state of reality we write,
\begin{equation*}
	\pi^\prime_t = (R^\prime_t-C^\prime_t)+\left(\frac{P^\prime_{t+1}}{1+r}-\frac{P^\prime_t(1+r)}{1+r}\right)\text{ for } t>0.
\end{equation*}
which can be simplified in the manner of \ref{pi1.1} by assuming competition returns profits to zero and markets function efficiently,
\begin{equation*}
\pi^\prime_t = (R^\prime_t-C^\prime_t)-P^\prime_t*\frac{r}{1+r}\text{ for } t>0.\label{pi2}
\end{equation*}

The difference between the treated and untreated states for the producer is zero in all subsequent periods due to the effects of competition and arbitrage, but in the period where the treatment occurs profits are non-zero,
\begin{equation}
	\Delta\pi_0=\frac{P^\prime_1-P_0}{1+r}=\frac{\Delta P_1}{1+r}.\label{pitzero}
\end{equation}
In the context of competitive markets for conventional goods like corn, the existence of profits would attract competition, boosting supply, leading to a lower equilibrium price of corn along with more production, more consumption, and more consumer surplus. However due to zoning and fundamental constraints to free entry in land markets, the profits are capitalized into the price of the real estate, and in most log-transformed empirical applications, the implicit prices of all bundled characteristics including the amenity in question increase.
Now we can calculate the present value of the change in total welfare due to the treatment,
\begin{eqnarray}
	\Delta W_{\mathrm{PV}}&=&\Delta \pi_0+\sum_{t>0}\Delta w_t(1+r)^{-t}\nonumber\\
	&=&\frac{\Delta P_1}{1+r}+\sum_{t>0} (\Delta \mathit{WTP}_t-\Delta R_t)(1+r)^{-t} \label{deltaW1}\\
	&=&\Delta P_{1_{\mathrm{PV}}}+\Delta\mathit{WTP}_\mathrm{PV}-\Delta R_{\mathrm{PV}}\label{deltaW2}
\end{eqnarray}
This is an intriguing result that helps refine how we compute the distribution of benefits from public projects that enhance amenities. The house price return is an unexpected profit enjoyed by the owner. This benefit to the firm is immediate and has a clear value, while the sign on the individual differences in the sum of future consumer benefits is indeterminate. the result in \ref{deltaW2} result may seem inconsistent with the usual measure of welfare change, such as in \cite{freeman14} and \cite{banzhaf20}, but next we consider the pattern of changing rent that accompanies the change in price to derive a more familiar measure. 

While $\Delta\pi_t=0$ for $t>0$, it is still helpful to consider the equation for the change in profits because it can help us understand how rents will change,
\begin{eqnarray}
	\Delta \pi_t &=& \Delta R_t-\Delta C_t-\Delta P_t\left(\frac{r}{1+r}\right)=0 \text{ for } t>0\nonumber\\
	\Delta R_t &=& \Delta C_t+\Delta P_t \left(\frac{r}{1+r}\right)  \text{ for } t>0\label{deltarent}
\end{eqnarray}
Returning to our measure of welfare change in equation \ref{deltaW1}, we can substitute for $\Delta R_t$  from \ref{deltarent} and rearrange,
\begin{eqnarray}
	\Delta W_{\mathrm{PV}}&=&\frac{\Delta P_1}{1+r}+\sum_{t>0} \left(\Delta \mathit{WTP}_t-\left(\Delta C_t+\Delta P_t \left(\frac{r}{1+r}\right)\right)\right)(1+r)^{-t} \nonumber\\
	%&=&\frac{\Delta P_1}{1+r}+\sum_{t>0} (\Delta \mathit{WTP}_t-\Delta C_t)(1+r)^{-t}-\Delta P_1 \sum_{t>0}  \left(\frac{r}{1+r}\right)(1+r)^{-t}\nonumber\\
	&=&\frac{\Delta P_1}{1+r}+\sum_{t>0} (\Delta \mathit{WTP}_t-\Delta C_t)(1+r)^{-t}-\frac{\Delta P_1}{1+r}\nonumber\\
	&=&\Delta \mathit{WTP}_{\mathrm{PV}}-\Delta C_{\mathrm{PV}} \label{deltaW2}
\end{eqnarray}
Which is the dynamic counterpart to the usual welfare measure (e.g., \cite{freeman14}). It is important to note that this measure does not convey the imbalanced flow of benefits to the owner and the renter. Indeed, because the public policy that improved the amenity was targeting consumers who would directly benefit, it is a pecuniary externality that competition among renters allows the homeowner to capture a large share of the benefits through rising rent that propels a rising asset value.

\subsection{Marginal Effects}
We are ultimately interested in understanding the relationship between price changes and welfare measures. In the usual approach that is static and does not distinguish between owner and renter, consumer optimization leads to a first order condition,$\frac{\partial{\mathit{WTP}}}{\partial q}=\frac{\partial P_h}{\partial q}$, which is the theoretical result that connects observable prices to WTP. We take a similar approach using the dynamic model described above, but with an infintesimally small treatment, $\partial q_{t>0}$ applied to all future time periods. We rely on the first order condition in equation \ref{foc} from the renter's optimal choice to reveal information about the consumer's marginal willingness to pay for $q$,

\begin{eqnarray}
\frac{\partial \mathit{WTP}_{\mathrm{PV}}}{\partial q_{t>0}}&=&\sum_{t>0} \frac{\partial \mathit{WTP}_t}{\partial q_t}(1+r)^{-t}\label{pvmwtp}\nonumber\\ 
&=&\sum_{t>0} \frac{\partial R_t}{\partial q_t}(1+r)^{-t}\nonumber\\
&=&\sum_{t>0} \frac{\partial C_t}{\partial q_t}(1+r)^{-t}+\frac{r}{1+r}\sum_{t>0} \frac{\partial P_t}{\partial q_t}(1+r)^{-t}\nonumber
\end{eqnarray}
Which can be rearranged to help us understand the partial derivative of the price function,
\begin{equation}
	\frac{\partial P_{1_{\mathrm{PV}}}} {\partial q_{t>0}}=\frac{\partial \mathit{WTP}_{\mathrm{PV}}}{\partial q_{t>0}}-\frac{\partial C_{\mathrm{PV}}}{\partial q_{t>0}} \label{marginalPrice}
\end{equation}
If we assume that costs do not change, then we have, in present value terms, the standard tangency condition used in the static ownership literature. If marginal costs are positive (e.g., increased property taxes), then the implicit price tends to underestimate marginal WTP. On the other hand, if marginal costs are negative (e.g., reduced defensive expenditures), then the implicit price tends to overestimate marginal WTP.

\subsection{Renters Who Own}
If we consider renters who also own a home (though not necessarily the one they rent), we can evaluate the effects on the market of increased wealth due to a shock (e.g., an unexpected policy or treatment) that increases profits. Below we use $\rho$ and $o$ superscripts to indicate the housing asset the individual rents or owns respectively. If we assume that increases in wealth due to profits from homeownership translate to smooth increases in consumption in perpetuity (implying the household's marginal propensity to consume (MPC) the perpetuity payment equals one), then consumption becomes a function of attributes of the rented and owned houses, such as $q^\rho$ and $q^o$. So we can define WTP for a renter who also own a house,
%\begin{eqnarray*}	
%	\mathit{WTP}_t^{\mathit{owner}}&=&\mathit{WTP}(q_t,z_t,y_t(q_t^o)) \text{ and} \\
%	\mathit{WTP'}_t^{\mathit{owner}}&=&\mathit{WTP}(q'_t,z_t,y_t(q'_t^o)) \text{ for } t>0\\
%	&=&	\mathit{WTP}(q'_t,z_t,y_t+\pi_0*r) \text{ for } t>0.
%\end{eqnarray*}
\begin{eqnarray*}
	\mathit{WTP}_t^{\prime\mathit{owner}}&=&\mathit{WTP}(q_t^{\prime \rho},z_t^\rho,y_t^\prime)\\
	&=&\mathit{WTP}(q_t^{\prime \rho},z_t^\rho,y_t+r*\pi^o_{\mathrm{PV}})
	%&=&\mathit{WTP}(q_t^{\prime \rho},z_t^\rho,y_t+\Delta \pi_0^o*r) \text{ for } t>0
\end{eqnarray*}
Where $r*\pi_{\mathrm{PV}}^o$ is the perpetual income payment from any profits associated with homeownership. Implicitly, the previous WTP measures are for renters who do not own a house.

Now we can consider the marginal welfare measure from an increase in the amenity at all relevant houses for a market that includes renter-owners. We start by defining a idiosyncratic and systematic component for each house's amenity, such that $q_{t}^i=\eta_{t}^i+\epsilon_t$, where the superscript, $i$, indexes the houses in a region. Next we consider a small shock to future levels of the systematic component of the amenity, $\epsilon_t$,% such that $\epsilon^\prime_t-\epsilon_{t}>0$ for $t>0$.  

\begin{eqnarray}
	\frac{\partial \mathit{WTP}^{\mathit{owner}}_{\mathrm{PV}}}{\partial \epsilon_{t>0}}&=&\sum_{t>0}\frac{\partial \mathit{WTP}_t}{\partial q^\rho_t}*(1+r)^{-t} \nonumber\\
	&& +\sum_{t>0}\frac{\partial \mathit{WTP}_t}{\partial y_t}*\frac{\partial y_t}{\partial q_t^o}(1+r)^{-t} \nonumber\\
	%&=&\sum_{t>0}\frac{\partial \mathit{WTP}_t^{\mathit{renter}}}{\partial q_t}*(1+r)^{-t}\nonumber\\
	%&& +\sum_{t>0}\frac{\partial \mathit{WTP}_t^{\mathit{renter}}}{\partial y_t}*\frac{\partial y_t}{\partial q_t}(1+r)^{-t} \nonumber
	&=&\sum_{t>0}\frac{\partial \mathit{WTP}_t}{\partial q_t^\rho}*(1+r)^{-t}\nonumber\\
	&& +\sum_{t>0}\frac{\partial \mathit{WTP}_t}{\partial y_t}*r*\frac{\partial P^o_{1_\mathrm{PV}}}{\partial q_t^o}(1+r)^{-t} \nonumber
\end{eqnarray}
Where $P^o_{1_\mathrm{PV}}$ is the present value of the house the individual owns at the start of $t=1$. Next we assume for the moment that the number of renter-owners is small, so their actions do not affect house prices. We also assume $\partial \mathit{WTP}_t/\partial y_t=\lambda$, where $\lambda$ is constant over time, allowing us to write,
\begin{eqnarray}
	\frac{\partial \mathit{WTP}^{\mathit{owner}}_{\mathrm{PV}}}{\partial \epsilon_{t>0}} 
	&=&\frac{\partial \mathit{WTP_{\mathrm{PV}}}}{\partial q^\rho_{t>0}} + \frac{\lambda}{1+r}*\frac{\partial P_1}{\partial q^o_{t>0}} \label{mwtpOwner1}.
\end{eqnarray}
The first term in the right hand side of \ref{mwtpOwner1} describes the renter's future optimizing decisions in the rental market, where choosing from housing options causes $q_t^\rho$ to vary, so we can solve equation \ref{marginalPrice} for marginal WTP and substitute it into the first term of \ref{mwtpOwner1}. Then we assume that the derivatives of the cost and price functions with respect to the amenity are the same for the house that the individual rents and the house they own, abstracting from the distinction between the two houses, 

\begin{eqnarray*}
	\frac{\partial \mathit{WTP}^{\mathit{owner}}_{\mathrm{PV}}}{\partial \epsilon_{t>0}} =	\frac{\partial C_{\mathrm{PV}}}{\partial q_{t>0}}+\frac{1+\lambda}{1+r}*\frac{\partial P_1}{\partial q_{t>0}}.
\end{eqnarray*}
Rearranging the result highlights the relationship between marginal willingness to pay and marginal sale price for permanent changes in an amenity,
\begin{equation*}
	\frac{\partial P_{1_{\mathrm{PV}}}} {\partial q_{t>0}}=\left(
	\frac{\partial \mathit{WTP}^{\mathit{owner}}_{\mathrm{PV}}}{\partial q_{t>0}}-\frac{\partial C_{\mathrm{PV}}}{\partial q_{t>0}}\right)\left(\frac{1}{1+\lambda}\right)
\end{equation*}


When the renter-owner is choosing the optimal level of $q_t$ by varying the house they rent each time period (i.e., sorting), there is no change in profits, so the tangency condition in equation \ref{foc} implied by a utility maximizing renter is not affected by ownership because the owner will be paid the same regardless of who lives in the house. However, if an increase in the amenity comes as a shock and affects all relevant houses, then the marginal price under-estimates the benefit of the amenity change to renter-owners. Conversely, we expect that in markets dominated by renter-owners, the marginal price over-estimates the benefits to renters who do not own. This also implies that clusters of owners may experience larger price changes as the wealth effects of increased WTP would translate to a cluster of higher prices.

Additionally, to the extent that the MPC is below one and households do accumulate wealth, we expect wealth effects to drive up rent and house prices in a pattern analogous to multiplier effects in short-run macroeconomic analysis (\cite{samuelson39}), particularly in locations where renter-owners tend to live and more so where wealth accumulates rapidly. If $\lambda$ increases with wealth, stratification would tend to be stronger.



\subsection{Taxes}
It can be useful to separate taxes out as a component of costs to consider how changes in property tax rates impact renters and owners. Assume firms pay a property tax, $T_t$ at the end of each time period on $P_{t+1}$, where to $T_t=g*P_{t+1}*\frac{1}{1+r}$. We can incorporate this into the firm's profit equation from \ref{pi1},
\begin{equation*}
	\pi^T_t = (R^T_t-C^T_t)+\left(\frac{P^T_{t+1}(1-g)}{1+r}-\frac{P^T_t(1+r)}{1+r}\right).\label{pi1T}
\end{equation*}
The tax is imposed at the end of $t=0$. Markets subsequently adjust to a new equilibrium with competition returning profits to zero and allowing us to express equilibrium rents as a function of the new tax,
\begin{equation*}
	R^T_t=C^T_t-\left(\frac{P^T_{t+1}(1-g)}{1+r}-P^T_t\right) \text{ for }t>0
\end{equation*}
Then efficient use of information and competition cause prices in the next period to equal prices in this period, allowing us to simplify the rental equation,
\begin{equation*}
R^T_t=C^T_t-\frac{P^T_{t}(g+r)}{1+r}\text{ for }t>0.
\end{equation*}
Because the tax does not influence decisions made at the margin, we assume rents and costs do not change due to the property tax. 


The firm's profits are driven by competition to zero in subsequent years, but in year 0 the firm's profits will be non-zero due to the adjustment in price after $t=0$ that is a consequence of the tax,
\begin{equation*}
	\pi^T_0 = (R^T_0-C^T_0)+\left(\frac{P^T_{1}(1-g)}{1+r}-P_0\right).\label{piT0}
\end{equation*}
Because of our assumption that rents and costs do not change due to the property tax, we can solve for the change in price due to the tax,
\begin{eqnarray}
	\Delta^T R_t&=&\Delta^T C_t+P^T_{t}\frac{g+r}{1+r}-P_{t} \frac{r}{1+r},\nonumber \\
	0&=&0+P^T_{t}\frac{g+r}{1+r}-P_{t} \frac{r}{1+r},\nonumber \\
	P^T_t&=& P_t\frac{r}{g+r}.\label{TxPrice}\nonumber
\end{eqnarray}
Where $\Delta^T$ denotes the difference between the state of reality with the tax and without. As long as g and r are positive, the property tax will cause the price of the house to fall to a new equilibrium level.
Next we can consider the change in profits during $t=0$ as a function of $g$ and $P_0$,
\begin{eqnarray*}
	\Delta^T \pi_0 &=& P_1^T\frac{1-g}{1+r}-P_1\frac{1}{1+r}\\
	&=& P_1\left(\frac{r(1-g)}{(g+r)(1+r)}-\frac{1}{1+r}\right)\\
	&=&-P_0\frac{g}{g+r}
\end{eqnarray*}
Where we have relied on equality of prices between the two scenarios in $t=0$ and also that without a tax $P_t=P_0$. The total welfare impact on the firm and household is simply this change in profits in time period zero.
Given that rents do not change but the interest payments on the price of the house decrease, we can decompose the change in rents due to the tax,
\begin{eqnarray*}
	\Delta^T R_t&=&P^T_t\frac{g+r}{1+r}-P_t\frac{r}{1+r}\\
	0&=&P^T_t\frac{g}{1+r}-(P_t-P_t^T)\frac{r}{1+r}.
\end{eqnarray*}
The first term is the present value of the annual tax payment and the second term is the rental payment reduction due to a lower house price. The two terms exactly offset each other.

\subsection{A Corrective Tax}
While pecuniary externalities are not an inefficiency that merits correction on the grounds of economic efficiency, political and ethical perspectives may call for such a correction. For example, a public expenditure that enhances air quality is likely designed to benefit the households in a region, regardless of whether or not they own a house. We are interested in identifying a property tax rate that can be designed to exactly offset the pecuniary externality that transfers a portion of the benefits of the amenity change from renter to owner.

Consider an increase in amenities that causes a change in welfare, as described in \ref{deltaW2}. Next, let $T_t$ for $t>0$ be a tax on the home owner, and let, $S_t$ for $t>0$, be an equivalent subsidy paid to the renter. We can incorporate these into the welfare measure in \ref{deltaW2},
\begin{eqnarray*}
	\Delta W_{\mathrm{PV}}&=& \left(\Delta P_{\mathrm{PV}}-T_{\mathrm{PV}}\right)+\left(\Delta\mathit{WTP}_\mathrm{PV}-\Delta R_{\mathrm{PV}}+S_{\mathrm{PV}}\right).\nonumber
\end{eqnarray*}
To offset the owner's unexpected profits due to the increased amenities in \ref{pitzero}, we set the tax equivalent to the change in profits, that come about due to the change in amenities,
\begin{eqnarray*}
 \Delta \pi_{\mathrm{PV}}&=&T_{\mathrm{PV}}\\
\frac{\Delta P_1}{1+r}&=&P_1\frac{g}{g+r}
\end{eqnarray*}
which can be solved for the corrective property tax rate, $g$,
\begin{equation}
	g=\frac{\% \Delta P_1*r}{1+r-\% \Delta P_1} \label{gcorrect}
\end{equation}
Because hedonic models are often estimated with a log-transformed dependent variable, estimating \ref{gcorrect} is straightforward, particularly for small changes in $q_{t>0}$ when regression coefficients approximate measures of percentage change. 

To avoid the situation where the firm captures the benefits of the subsidy, the payment of the subsidy would need to be independent of the renter's decision about where to live. Renter-owners would experience no price change and they would receive a subsidy as well as pay a tax. Firms that own multiple houses would pay the tax multiple times while households receive all of the subsidies. Firms would experience no change in profits because the pecuniary impact of the corrective tax would cancel out the pecuniary impact of the amenity improvement.
Notably, this policy is revenue neutral for the government and the firm because, 
\begin{equation*}
	P_0*\frac{g}{g+r}=T_{\mathrm{PV}}=S_{\mathrm{PV}}=\Delta \pi_{\mathrm{PV}}=-\Delta^T \pi_{\mathrm{PV}}=\sum_{t>0}\frac{g*P_t^T}{(1+r)^{-t}}.
\end{equation*}
Where the first term is the change in profits from the tax and the last term is the government's tax revenue. 

 
\section{Empirical Application}
Empirical applications of the hedonic price method that identify measures of willingness to pay can be used to quantify the tax rate, $g$, that generates tax revenue $T_t$ and subsidy $S_t$. Because logarithmic transformations of the dependent variable, the house price, are standard in empirical applications, it is straightforward to estimate $g$ for a known value of $\%\Delta P_1 $. 

Additionally, quantification of the additional marginal benefits experienced by renter-owners due to the income effects requires relatively little data if an existing hedonic price study estimates a coefficient for the income of the occupant that can be used as an estimate of $\lambda$.

\cite{chaygreenstone05} quantified the benefits to households of increased air quality, using non-attainment status under the Clean Air Act Amendments as an exogenous instrument for quantifying the causal impact of total suspended particulates (TSPs) on housing values. They find that the amendments decreased TSP concentrations by about 10 $\mu g/m^3$.They also estimated that housing prices increase by about 0.28 percent for each $1 \text{ }{\mu g/m^3}$ decrease in TSP concentrations. Then they assume constant marginal WTP and estimate the willingness to pay for the improvements . \cite{chaygreenstone05} go on to discuss how the resulting value represents both an estimate of willingness to pay and an estimate of the increased price of houses. We would add to that list that this value estimates the present value of the increase in future rental prices that consumers will pay for cleaner air.

\cite{bishop19} apply their model of a forward looking consumer to the same data as \cite{chaygreenstone05}, and develop a correction factor for scaling marginal willingness to pay that ranges from $\frac{1}{0.519} \approx 1.93$ to $\frac{1}{0.640} \approx 1.56$. They use this scaling factor to adjust the estimates of average WTP found by \cite{chaygreenstone05} away from equality with the sale price based on the assumption that  
\cite{smithhuang95} estimate parameters for income 
...

...






\section{Discussion}
Throughout this paper we have mostly ignored the methodological challenges associated with quantifying changes in WTP that dominate the literature. It seems likely to us that the way future willingness to pay, future rents, future costs, etc are capitalized into today's prices increase the challenge associated with identifying an individual's demand. At the same time, we expect that the enhanced clarity that comes from separating the renter from the owner will ultimately determine the appropriate technique for fully quantifying welfare change.

The scope of a property owner's legal rights is an important component of evaluating the feasibility of applying a corrective tax like we describe above. For example, even if tenants have a human right to the health benefits of clean air, it is less clear that tenants are entitled to all benefits of clean air, such as reduced maintenance costs from lower levels of suspended particulates (\cite{bajari12}). 

We emphasize that the capitalized values related to housing, e.g., the price of the house, are the result of expected interactions between households and firms with probabilities of buying/selling and renting/moving that are generally strictly between zero and one. The non-zero probability of moving that renters face each time period is built into consumer decision making and is thus reflected by housing prices. In the context of horizontal sorting equilibrium models for occupants of housing, the expectations of firms and households about this process are built into observed prices. Our simple model omits this information, which we expect would lead to underestimation of benefits due to underestimated costs such as moving costs. It would be interesting to extend our model to include forward looking renters to better understand the impacts of frequent moving on renters (\cite{bishop19}) 

The effects on house prices and owner well-being identified in this paper suggest an import consideration in the development of sorting models such as the single cross (\cite{banzhaf20}). Enhanced amenity values in a locality, may lead to re-sorting because the affected house are relatively more valuable, a re-ordering of houses. Simultaneously, the sorting order of the households will change if some of them are owners too, as their willingness to pay may rise enough to change the ordering of households. Furthermore, if higher income households tend to 

A key assumption behind our use of the term, pecuniary externality, is that the policy in consideration is targeting households not property owners. While property taxes are familiar, subsidies targeting renters are uncommon, likely because they would effectively be an amenity that would be captured by the owner. A lump sum transfer to all resident in the affected region of the total tax receipt divided across all houses would be one approach to paying out the subsidy to renters. Renters in low-rent housing, who are likely to have a high marginal utility of income and a high marginal propensity to consume would also receive a larger subsidy than their landlords would pay in property taxes, a potential path toward restorative justice. 

Considering the spatial correlations that exist between poverty and pollution (\cite{banzhafJustice19}), environmental cleanup and various publicly provided services, the designers of the subsidy would have numerous opportunities to target payments at individuals with a high marginal utility of income.

This research does not address some fundamental limitations of economic measures of well-being. The WTP measures available for revealed preference analysis are based on a households budget or ability to pay. 

An important limitation of this work is the absence of investment in the model, which we expect would be sensitive to property taxes, leading to a trade-off in well-being for residents of an area that institute a corrective tax as developed above. We expect the optimal tax would be lower than presented here when the incentives to invest in new housing are taken into account. 

While we have focused on single family homes in the development of our model, these results are applicable to any productive asset, such as a factory. For example, if an industry uses natural resources in production, then public enhancements to natural resources will tend to reduce production costs increasing profitability if competition . 

Extending the work of \cite{kanemoto88} to distinguish between renter-owners and renters in a dynamic, general equilibrium framework would be another interesting extension of this work.
\begin{thebibliography}{}
%

\bibitem[\protect\citeauthoryear{Bajari et. al.}{2012}]{bajari12}
\textsc{Bajari, P., Fruehwirth J.C., Kim, K.I., Timmins, C.} (2012):
`` A Rational Expectations Approach to Hedonic Price Regressions with Time-Varying Unobserved Product Attributes: The Price of Pollution''
\textit{American Economic Review}, 102(5), 1898--1926.
\endbibitem

\bibitem[\protect\citeauthoryear{Banzhaf}{2020}]{banzhaf20}
\textsc{Banzhaf, S.} (2020):
``Panel Data Hedonics Rosen’s First Stage as a “Sufficient
Statistic",''
\textit{International Economic Review}, 61(2), 973--1000.
\endbibitem


\bibitem[\protect\citeauthoryear{Banzhaf, Ma, and Timmins}{2019}]{banzhafJustice19}
\textsc{Banzhaf, S., Ma, L. and Timmins, C.} (2019):
``Environmental Justice: The Economics of Race, Place, and Pollution,''
\textit{Journal of Economic Perspectives}, 33(1), 185--208.
\endbibitem

\bibitem[\protect\citeauthoryear{Bishop, et al.}{2020}]{bishop20}
\textsc{Bishop, K.C., Kuminoff, N.V., Banzhaf, H.S., Boyle, K.J., von Gravenitz, K., Pope, J.C., Smith, V.K. and Timmins, C.D., } (2020):
``Best practices for using hedonic property value models to measure willingness to pay for environmental quality,''
\textit{Review of Environmental Economics and Policy}, 14(2), 260--281.
\endbibitem

\bibitem[\protect\citeauthoryear{Bishop and Murphy}{2011}]{bishop11}
\textsc{Bishop, K.C. and Murphy, A.D.} (2011):
``Estimating the Willingness to Pay to Avoid Violent Crime: A Dynamic Approach,''
\textit{American Economic Review}, 101(3), 625--629.
\endbibitem

\bibitem[\protect\citeauthoryear{Bishop and Murphy}{2019}]{bishop19}
\textsc{Bishop, K.C., and Murphy, A.D.} (2019):
``Valuing time-varying attributes using the hedonic model: when is a dynamic approach necessary?,''
\textit{Review of Economics and Statistics}, 101(1), 134--145.
\endbibitem

\bibitem[\protect\citeauthoryear{Bishop and Timmins}{2018}]{bishoptimmins18}
\textsc{Bishop, K.C., and Timmins, C.} (2018):
``Using Panel Data to Easily Estimate Hedonic Demand Functions,''
\textit{Journal of the Association of Environmental and Resource Economists}, 5(3), 517--543.
\endbibitem

\bibitem[\protect\citeauthoryear{Buchanan and Stubblebine}{1962}]{bs1962}
\textsc{Buchanan, J.M. and Stubblebine, W.C.} (1962):
``Externality,''
\textit{Economica}, 29(116), 371-384.
\endbibitem

\bibitem[\protect\citeauthoryear{Bui and Mayer}{2003}]{bui03}
\textsc{Bui, L.T. and Mayer, C.J.} (2003):
``Regulation and capitalization of environmental amenities: evidence from the toxic release inventory in Massachusetts,''
\textit{Review of Economics and statistics}, 85(3), 693--708.
\endbibitem


\bibitem[\protect\citeauthoryear{Chay and Greenstone}{2005}]{chaygreenstone05}
\textsc{Chay, K.C. and Greenstone, M.} (2005):
``Does Air Quality Matter? Evidence from the Housing Market,''
\textit{The Journal of Political Economy}, 113(2), 376--424.
\endbibitem
 
\bibitem[\protect\citeauthoryear{Dávila and Korinek}{2018}]{davila18}
\textsc{Dávila, E. and Korinek, A.} (2018):
``Pecuniary externalities in economies with financial frictions,''
\textit{The Review of Economic Studies}, 85(1), 352--395.
\endbibitem

\bibitem[\protect\citeauthoryear{Freeman}{1974}]{freeman74}
\textsc{Freeman, A.M.} (1974):
``On Estimating Air Pollution Control Benefits from Land Value Studies,''
\textit{Journal of Environmental Economics and Management}, 1(1), 74--83.
\endbibitem

\bibitem[\protect\citeauthoryear{Freeman}{1980}]{freeman80}
\textsc{Freeman, A.M.} (1980):
``Land Prices Substantially Underestimate the Value of Environmental Quality: A Comment,''
\textit{The Review of Economics and Statistics}, 62(1), 154--156.
\endbibitem


\bibitem[\protect\citeauthoryear{Freeman}{1999}]{freeman99}
\textsc{Freeman, A.M.} (1999):
\textit{The Measurement Of Environmental And Resource Values: Theory And Methods}.
Washington DC, U.S.A.: Resources for the Future.
\endbibitem
 
\bibitem[\protect\citeauthoryear{Freeman, Herriges, and Kling}{2014}]{freeman14}
\textsc{Freeman III, A.M., Herriges, J.A., and Kling, C.L.} (2014):
\textit{The Measurement Of Environmental And Resource Values: Theory And Methods}.
Washington DC, U.S.A.: Routledge
\endbibitem 

\bibitem[\protect\citeauthoryear{Hendershott and Slemrod}{1983}]{hendershott83}
\textsc{Hendershott, P.H. and Slemrod, J.} (1983):
``Taxes and the User Cost of Capital for Owner-Occupied Housing,''
\textit{Real Estate Economics}, 10(4), 375--393.
\endbibitem

\bibitem[\protect\citeauthoryear{Holcombe and Sobel}{2001}]{holcombe01}
\textsc{Holcombe, R.G. and Sobel, R.S.} (2001):
``Public Policy Toward Pecuniary Externalities,''
\textit{Public Finance Review}, 29(4), 304--325.
\endbibitem

\bibitem[\protect\citeauthoryear{Kanemoto}{1988}]{kanemoto88}
\textsc{Kanemoto, Y.} (1988):
``Hedonic Prices and the Benefits of Public Projects,''
\textit{Econometrica}, 56(4), 981--989.
\endbibitem

\bibitem[\protect\citeauthoryear{Kuminoff and Pope}{2014}]{kuminoffpope14}
\textsc{Kuminoff, N.V., and Pope, J.C.} (2014):
``Do Capitalization Effects Measure the Willingness to Pay for Public Goods?  .,''
\textit{International Economic Review}, 55(4), 1227--1250.
\endbibitem

\bibitem[\protect\citeauthoryear{Niskanen and Hanke}{1977}]{niskanen77}
\textsc{Niskanen, W.A. and Hanke, S.H.} (1977):
``Land prices substantially underestimate the value of environmental quality,''
\textit{The Review of Economics and Statistics}, 59(3), 375--377.
\endbibitem

\bibitem[\protect\citeauthoryear{Palmquist}{1984}]{palmquist84}
\textsc{Palmquist, R.B.} (1984):
``Estimating the Demand for the Characteristics of Housing,''
\textit{The Review of Economics and Statistics}, 66(3), 394--404.
\endbibitem


\bibitem[\protect\citeauthoryear{Palmquist}{1989}]{palmquist89}
\textsc{Palmquist, R.B.} (1989):
``Land as a Differentiated Factor of Production: A Hedonic Model and its Implications for Welfare Measurement,''
\textit{Land Economics}, 65(1), 23--28.
\endbibitem

\bibitem[\protect\citeauthoryear{Poterba}{1984}]{poterba84}
\textsc{Poterba, J. M.} (1984):
``Tax Subsidies to Owner-Occupied Housing: An Asset-Market Approach,''
\textit{The Quarterly Journal of Economics}, 99(4), 729--752.
\endbibitem

\bibitem[\protect\citeauthoryear{Rosen}{1974}]{rosen74}
\textsc{Rosen, S.} (1974):
``Hedonic Prices and Implicit Markets: Product Differentiation in Pure Competition,''
\textit{The Journal of Political Economy}, 82(1), 34--55.
\endbibitem

\bibitem[\protect\citeauthoryear{Samuelson}{1939}]{samuelson39}
\textsc{Samuelson, P.} (1939):
``Interactions between the Multiplier Analysis and the Principle of Acceleration,''
\textit{The Review of Economics and Statistics}, 62(2), 75--78.
\endbibitem

\bibitem[\protect\citeauthoryear{Scitovsky}{1954}]{scitovsky54}
\textsc{Scitovsky, T.} (1954):
``Two Concepts of External Economies,''
\textit{Journal of political Economy}, 21(2), 143--151.
\endbibitem


\bibitem[\protect\citeauthoryear{Smith and Huang}{1995}]{smithhuang95}
\textsc{Smith, V.K. and Huang, J.C.} (1995):
``Can Markets Value Air Quality? A Meta-Analysis of Hedonic Property Value Models,''
\textit{Journal of Political Economy}, 103(1), 209--227.
\endbibitem


\bibitem[\protect\citeauthoryear{Sonstelie and Portney}{1980}]{sonstelie80}
\textsc{Sonstelie, J. C., and Portney, P. R.} (1980):
``Gross rents and market values: testing the implications of Tiebout's hypothesis,''
\textit{Journal of Urban Economics}, 7(1), 102-118.
\endbibitem

\bibitem[\protect\citeauthoryear{Zabel and Kiel}{2000}]{zabelkiel00}
\textsc{Zabel, J.E., Kiel, K.E.} (2000):
``Estimating the Demand for Air Quality in Four U.S. Cities,''
\textit{Land Economics}, 76(2), 174--194.
\endbibitem

\bibitem[\protect\citeauthoryear{Zilberman, et al.}{2013}]{zilberman13}
\textsc{Zilberman, D., Barrows, G., Hochman, G. and Rajagopal, D.} (2013):
``On the Indirect Effect of Biofuel,''
\textit{American Journal of Agricultural Economics}, 95(5), 1332--1337.
\endbibitem

\bibitem[\protect\citeauthoryear{Zilberman, Hochman, and Rajagopal}{2011}]{zilberman11}
\textsc{Zilberman, D., Hochman, G. and Rajagopal, D.} (2011):
``On the Inclusion of Indirect Land Use in Biofuel,''
\textit{University of Illinois Legal Review}, 2011(2), 413--434.
\endbibitem

%example references
%
%\bibitem[\protect\citeauthoryear{Aumann}{1987}]{b1}
%\textsc{Aumann, R. J.} (1987):
%``Correlated Equilibrium as an Expression of Bayesian Rationality,''
%\textit{Econometrica}, 55, 1--18.
%\endbibitem
%
%\bibitem[\protect\citeauthoryear{Peck}{1994}]{b2}
%\textsc{Peck, J.} (1994):
%``Competition in Transactions Mechanisms: The Emergence of Competition,''
%Unpublished Manuscript, Ohio State University.
%\endbibitem
%
%\bibitem[\protect\citeauthoryear{Enelow and Hinich}{1990}]{b3}
%\textsc{Enelow, J., and M. Hinich}, eds. (1990):
%\textit{Advances in the Spatial Theory of Voting}.
%Cambridge, U.K.: Cambridge University Press.
%\endbibitem
%
%\bibitem[\protect\citeauthoryear{Wittman}{1990}]{b4}
%\textsc{Wittman, D.} (1990):
%``Spatial Strategies when Candidates Have Policy Preferences,''
%in \textit{Advances in the Spatial Theory of Voting},
%ed. by M. Hinich and J. Enelow.
%Cambridge, U.K.: Cambridge University Press, 66--98.
%\endbibitem
%
%\bibitem[\protect\citeauthoryear{Cahuc, Postel-Vinay and Robin}{2006}]{b5}
%\textsc{Cahuc, P., F. Postel-Vinay, and J.-M. Robin} (2006): 
%``Supplement to `Wage Bargaining with On-the-Job Search: Theory and Evidence',''
%\textit{Econometrica Supplementary Material}, 74.
%\endbibitem
\end{thebibliography}

\end{document}
