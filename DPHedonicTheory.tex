% use option [draft] for initial mission
%            [final] for the prepublication
\documentclass[ecta,nameyear,draft]{econsocart}
%
%\usepackage{}
\RequirePackage[colorlinks,citecolor=blue,linkcolor=blue,urlcolor=blue,pagebackref]{hyperref}

\startlocaldefs

%%%%%%%%%%%%%%%%%%%%%%%%%%%%%%%%%%%%%%%%%%%%%%
%%                                          %%
%% Uncomment next line to change            %%
%% the type of equation numbering           %%
%%                                          %%
%%%%%%%%%%%%%%%%%%%%%%%%%%%%%%%%%%%%%%%%%%%%%%
%\numberwithin{equation}{section}
%%%%%%%%%%%%%%%%%%%%%%%%%%%%%%%%%%%%%%%%%%%%%%
%%                                          %%
%% For Assumption, Axiom, Claim, Corollary, %%
%% Lemma, Theorem, Proposition, Hypothezis, %%
%% Fact                                     %%
%% use \theoremstyle{plain}                 %%
%%                                          %%
%%%%%%%%%%%%%%%%%%%%%%%%%%%%%%%%%%%%%%%%%%%%%%
\theoremstyle{plain}
\newtheorem{axiom}{Axiom}
\newtheorem{claim}[axiom]{Claim}
\newtheorem{theorem}{Theorem}[section]
\newtheorem{lemma}[theorem]{Lemma}
\newtheorem*{fact}{Fact}
%%%%%%%%%%%%%%%%%%%%%%%%%%%%%%%%%%%%%%%%%%%%%%
%%                                          %%
%% For Definition, Example, ,         %%
%% Notation, Property                       %%
%% use \theoremstyle{}                %%
%%                                          %%
%%%%%%%%%%%%%%%%%%%%%%%%%%%%%%%%%%%%%%%%%%%%%%
\theoremstyle{remark}
\newtheorem{definition}[theorem]{Definition}
\newtheorem*{example}{Example}

%%%%%%%%%%%%%%%%%%%%%%%%%%%%%%%%%%%%%%%%%%%%%%
%% Please put your definitions here:        %%
%%%%%%%%%%%%%%%%%%%%%%%%%%%%%%%%%%%%%%%%%%%%%%


\endlocaldefs

\begin{document}

\begin{frontmatter}

\title{Public Goods, Pecuniary Externalities, and Hedonic Assets}
\runtitle{Hedonic Assets}

\begin{aug}
% use \particle for den|der|de|van|von (only lc!)
% [id=?,addressref=?,corref]{\fnms{}~\snm{}\ead[label=e?]{}\thanksref{}}
%
%% e-mail is mandatory for each author
%
%%% initials in fnms (if any) with spaces
%
\author[id=au1,addressref={add1}]{\fnms{Douglas}~\snm{Patton}\ead[label=e1]{patton.douglas@epa.gov}}
%\author[id=au2,addressref={add2}]{\fnms{Second}~\snm{Author}\ead[label=e2]{second@somewhere.com}}
%\author[id=au3,addressref={add2}]{\fnms{Third}~\snm{Author}\ead[label=e3]{third@somewhere.com}}
%%%%%%%%%%%%%%%%%%%%%%%%%%%%%%%%%%%%%%%%%%%%%%
%% Addresses                                %%
%%%%%%%%%%%%%%%%%%%%%%%%%%%%%%%%%%%%%%%%%%%%%%
\address[id=add1]{%
\orgdiv{ORISE Fellow participating at the US EPA},
\orgname{Office of Research and Development}}


\end{aug}

%% Put support info here.  Reminder: do not thank the handling coeditor anonymously or by name
\support{This research was supported in part by an appointment to the ORISE Fellowship Program at the U.S. EPA, Office of Research and Development, Athens, GA, administered by the Oak Ridge Institute for Science and Education through Interagency Agreement No. DW8992298301 between the U.S. Department of Energy and the U.S. Environmental Protection Agency. This article has been reviewed by the U.S. Environmental Protection Agency and approved for publication. Mention of trade names or commercial products does not constitute endorsement or recommendation for use by the U.S. Government. The views expressed in this article are those of the authors and do not necessarily reflect the views or policies of the U.S. EPA.  %The authors would like to thank internal and journal peer reviewers for many useful comments on the manuscript. All remaining errors are my own.
}
%
\begin{abstract} %(restat - 100 words)
I develop a dynamic welfare model of a hedonic housing asset and an associated flow of housing services that are traded in closely linked asset and rental markets. The model uses a households willingness to pay rent to reveal their preferences while allowing for welfare calculations as a function of the house's sale price, providing important details about the distribution of benefits over time between renters and owners. I also derive a stylized tax and subsidy to offset the profits owners derive from renters’ willingness to pay for improved public goods. I apply the model to an existing hedonic study of the Clean Air Act Amendments to highlight the typically overlooked transfers between renters and owners. The model applies to a variety of productive assets.
\end{abstract}

\begin{keyword}
\kwd{Hedonic Price Method}
\kwd{Property Tax Incidence}
\kwd{Environmental Justice}
\kwd{JEL:Q51, Q56, Q57, Q58, D62, D63}
\end{keyword}

\end{frontmatter}
%%%%%%%%%%%%%%%%%%%%%%%%%%%%%%%%%%%%%%%%%%%%%%%%%%%%%%%%%%%%%%%%%%%%%%%%%
%%%% Main text entry area:
%%%%%%%%%%%%%%%%%%%%%%%%%%%%%%%%%%%%%%%%%%%%%%%%%%%%%%%%%%%%%%%%%%%%%%%%%

\section{Introduction}
\caps{The valuation tools} that economists have developed for quantifying ecosystem service benefits provide society with important information about human well-being. Hedonic pricing methods allow economists to quantify diverse ecosystem service values by modeling how people respond to trade-offs associated with observable market activity, typically in labor or housing markets. Information about the benefits that people derive from their homes is a vital component of public discourse and policymaking. In this paper, I revisit the economic theory of the hedonic price method as it is applied to residential real estate. The model I develop retains two important distinctions absent from existing models: between renter and owner as well as between housing as an asset with a sale price and housing as a service with a rental price. The distinction in the model between home owners and home renters highlights the enormous magnitude of value transfers from renters to owners that are typically obscured by existing applications of the hedonic price method to housing.

%\caps{Economic assessments of} ecosystem service values frequently focus on quantifying the direct economic benefits to households or consumers of changes to ecosystems. Revealed preference techniques such as the hedonic price method quantify ecosystem service benefits by using information about observed market prices paid by consumers for either a privately traded ecosystem service or a privately traded good consumed alongside public ecosystem services. Because of the connection between ecosystem services and private markets, ecosystem services benefit both consumers and firms. Additionally, applications of these techniques to residential housing services utilize asset prices rather than rental prices, a 

A key insight from this model is that conflating the housing rental market with the housing asset market conceals the distribution of benefits between home owners and home renters. The consequence of this conflation is that the net economic benefits of public projects are enjoyed to a much greater extent by the asset owners relative to asset renters than standard analyses typically imply. \cite{kuminoffpope14} say in their opening paragraph, "Homebuyers implicitly purchase the right to consume a bundle of local public goods when they buy a house." In the context of the present analysis I would rephrase this as, "Homebuyers implicitly purchase the right to \textit{rent out} a bundle of local public goods when they buy a house."

I contribute to the literature on the hedonic price method applied to sellable assets by developing a dynamic model that includes competitive equilibria in an asset market and a closely related rental market. Following the approach taken in national income accounting, I treat households as having potentially two roles: as consumers who rent housing and as firms who can own and rent-out housing. The dynamic modeling approach emphasizes the adjustments that follow shocks affecting residential real estate such as unexpected improvements in environmental amenities. Importantly, I develop a theoretical model that uses equilibrium in the rental market to reveal consumer preferences for housing amenities, which is in contrast to the common static approach that uses the price of the housing asset. Nonetheless, I derive measures of total welfare change based on the price of the housing asset that are consistent with the static literature (e.g., \cite{freeman99} and \cite{freeman14}). In the existing literature, much attention has been paid to estimating households' willingness to pay for improvements in environmental amenities as owner-occupants, though this assumption is often implicit. The increased granularity of my model highlights the distributional consequences of improving hedonic assets such as housing, particularly for the approximately one third of households in the United States who rent and do not own. 

The connectedness of the rental and asset markets demonstrated in my modeling approach can be thought of as examples of pecuniary externality. And economists arguably tend to de-emphasize these types of spillovers from their analysis (\cite{holcombe01}). As discussed below, the hedonic price method relies on changes in housing sale prices to quantify the economic values that occupants place on environmental amenities. The fact that the unit of housing as a productive asset has increased in value at (and because of) the expense of future renters due to the improved amenities is generally ignored (e.g., \cite{bishop20}) or glossed over (e.g., \cite{banzhaf20}) when assessing ecosystem service benefits. The dynamic model I develop allows economists to clearly separate the impacts on firms from the impacts on households; this distinction adds clarity to concerns about environmental justice that arise when public policies enhance property values, potentially leading to increased inequality and environmental gentrification (\cite{banzhafJustice19}). 

I also develop extensions for households that own houses and for property taxes. My results agree with existing dynamic models of property taxes in the literature, such as \cite{freeman80} and \cite{poterba84}, but with an entirely different motivation: environmental justice. I apply the model of property taxes to identify a  property tax rate and corresponding subsidy to consumers that affords policy makers options for increasing environmental quality without harming current residents by allowing firms to capture these benefits as profits. I also include a brief empirical application of the model to demonstrate the magnitude of the welfare transfers associated with the environmental benefits of the Clean Air Act Amendments. 

In the context of cost-benefit analysis and public policy, this paper establishes a tax and subsidy that can be used to make a potential pareto improvement a pareto improvement, though future research is clearly required. More importantly the dynamic approach makes more apparent the role of future renters in determining today's house prices. Unbiased cost-benefit analyses cannot simply assess how a policy impacts prices of affected properties regardless of the future occupants. Current residents may move away to escape rising rents and their decreased well-being is not be reflected by changes in house prices. Additionally, the new residents have welfare changes based on the difference between their old residence and the new one. Both of these effects require more data to measure and both effects reinforce an upward bias in welfare estimates. In the empirical application, I do not correct for this bias.


\section{Literature Review}

\subsection{Benefits of Amenities}

\cite{rosen74} and \cite{freeman74} contemporaneously established a static theoretical relationship between a household's utility and a hedonic price function for residential housing. \cite{freeman99} and more recently \cite{freeman14} are common references that continue to use the same theoretical justification for evaluating welfare changes. These authors found that hedonic price regressions were useful for estimating economic benefits because they revealed the implicit marginal price of and a household's willingness to pay for each of the bundled services provided by a house. 

The theoretical models developed by \cite{rosen74} and \cite{freeman74} treat each housing purchase as "pure consumption" (\cite{rosen74}), which makes the approach difficult to apply to renters who do not own a house. In their recent paper detailing best practices for hedonic assessments of ecosystem service values, \cite{bishop20} suggest a preference for rental-rate data where the proportion of owner-occupants is low, but they otherwise do not discuss the issue. In empirical applications, one option is to restrict analysis to owner-occupants. This approach is often followed by authors who seek to identify the homeowner's compensated demand curve. Empirical examples include \cite{palmquist84}, \cite{zabelkiel00}, and \cite{chaygreenstone05}. \cite{smithhuang95} focus their widely cited meta-analysis of hedonic valuation studies of clean air on owner-occupants. \cite{kuminoffpope14} include controls for "block group level ... percent owner occupied", but do not address how the measure might affect MWTP; \cite{chaygreenstone05} and \cite{bento15} include similar controls in their regression models.
 
A number of recent studies do not address the distinction between owners and renters when reconstructing consumer demand curves. \cite{bishoptimmins18} do not address the question of ownership and discuss only home buyers, without mention of the occupants. \cite{banzhaf20} mentions distinctions between landlords and renters when approximating demand functions, but as with \cite{freeman14} those differences cancel during aggregation of the benefits to consumers and firms and receive no further consideration. \cite{banzhaf21} considers only owner-occupied housing without explanation. Alternatively, due to the microeconomic and econometric challenges to identifying demand curves, as pointed out in \cite{bishoptimmins18}, a number of studies only seek to estimate implicit prices as estimates of marginal willingness to pay (MWTP) and any distinction between owners and renters is ignored (e.g., \cite{bajari12}, \cite{bui03}, and \cite{kuminoffpope14}).

A variety of theoretical dynamic models have been developed for hedonic analysis of real estate prices. These models tend to include more realistic treatment of taxation and interest rate issues but they lack a theoretical connection to a dynamic counterpart to MWTP. Early on, \cite{niskanen77} and \cite{freeman80} develop and extend dynamic hedonic models to quantify benefits in the presence of taxes. Around the same time, \cite{sonstelie80} develop a dynamic approach to hedonic modeling that incorporates interest rates and tax rates and the user costs a landlord experiences. \cite{hendershott83} distinguish between owners and renters when modeling the impacts of tax treatment on households. \cite{poterba84} take a similar approach and use the concept of user cost to assess how inflation and tax rates impact housing prices and investment in new construction in the context of macroeconomic asset markets, limiting their model to owner-occupied housing. More recently, \cite{bishop11} and \cite{bishop19} estimate dynamic models that include a consumer as a home-buyer's expectations about future prices. None of these works distinguish between the welfare of the renter and the owner of a housing unit, though \cite{hendershott83} distinguish between the income of a renter and an owner. 





\subsection{External Economies} 

In Economics, pecuniary and technological externalities have been defined to characterize interactions among market participants as well as political participants (\cite{holcombe01}). \cite{scitovsky54} discussed external economies in the context of industrial organization, distinguishing between technological and pecuniary externalities. \cite{buchananstubblebine} generalized technological externalities to include consumers. Technological externalities have a direct impact on an external party other than through prices; the numerous, rippling consequences of price changes are pecuniary externalities. \cite{holcombe01} suggest that because pecuniary externalities do not result in economic inefficiency, the associated losses and gains have received much less attention than non-pecuniary externalities. The direct mentions of pecuniary externalities in recent literature (e.g., \cite{davila18}) are largely confined to macroeconomic discussions related to investment and finance. 

The relatively rare investigations of pecuniary externalities in environmental economics often focus on indirect policy effects. The debate over the indirect effects of biofuel subsidies provides a complex example of the deliberations over the ways relative price changes may be included in policy design. \cite{zilberman11} examine the impact of biofuel subsidies on indirect land use change through increased food prices, and they begin by noting the validity of considering pecuniary externalities in policy design. In \cite{zilberman11} and other studies such as \cite{zilberman13} a similar group of authors cast doubt on the inclusion of indirect land use change in biofuel subsidy planning largely because of the myriad other indirect effects also left out.



\subsection{Environmental Justice} 
\cite{mohai09} broadly review the history of Environmental Justice, including both academic attention and societal attention to the topic. Much of the focus is on the intersection of race, poverty, and toxic waste disposal. Several of the sources they review illustrate a growing awareness of pollution and a consequent push by more affluent, whiter populations to maintain their own local environmental quality.  \cite{aaronson2021} discuss redlining and the consequent lack of access to credit and disinvestment in black neighborhoods that began in the 1930's and which continue to negatively affect a variety of measures of well-being for occupants of those neighborhoods. \cite{gervais2002} found that favorable tax treatment of homeowners encourages them to over consume housing at the expense of the stock of rental housing. This evidence suggests that favorable treatment of homeowners relative to landlords in the tax code may compound the negative effects of low home ownership rates.

In the context of applying the hedonic price method to residential real estate, environmental justice concerns and the search for causal connections face many challenges due to the numerous connections and feedback effects, as exemplified by the pyramid of environmental gentrification (\cite{banzhafJustice19}). \cite{banzhafJustice19} provide an overview of environmental justice in a spatial context that focuses on the use of hedonic price modeling for revealing preferences and disentangling the mechanisms that expose lower income households to higher levels of pollution. The same authors also recognize that improved environmental quality can lead to gentrification, displacing lower income households that tend to rent rather than own their homes. 

Increased public attention to environmental justice is evidenced by the recent executive order 14008 of Jan 27, 2021. The order focuses on communities as the target of restorative justice, suggesting that concerns about environmental gentrification should be reflected in policy design. \cite{curran12} describe a "just green enough" approach to enhancements in environmental quality that lead to reduced environmental gentrification by targeting enhancements at current residential and industrial occupants rather than new developments that tend to be "parks, cafes, and a riverwalk". \cite{wolch14} advocate for increased adoption of "just green enough" approaches to public enhancements of environmental quality after reviewing the literature on environmental gentrification in the US and China and concluding that displacement of working-class communities is a broadly relevant, ongoing concern. 

\subsection{Property Taxes}
Property taxes paid by the owners of residential properties to local municipalities provide the bulk of revenue to local governments (\cite{zodrow01}). For homeowners, these taxes have been found to be the most salient and among the most disliked (\cite{cabralhoxby}). \cite{tiebout56} provides an early and often cited theoretical sorting model connecting property tax rates and public benefits associated with diverse municipalities that levy taxes and provide services and diverse homeowners who choose where to live and thus which services to consumer and pay for, leading to an economically efficient outcome.

The economic literature on property taxes has focused considerable attention on the unresolved question of tax incidence (\cite{zodrow01} and \cite{sirmans08}). The property tax capitalization literature has sought to resolve these questions by determining if property taxes and public services are fully capitalized into housing prices. Economic theory suggests a variety of barriers to achieving the necessary econometric identification for assessing this capitalization question including the following: housing supply elasticities, the mobility of housing capital, and the mobility of households (\cite{sirmans08}). \cite{england16} provides a detailed overview of the evolution of the debate on how tax incidence is distributed between tenants and owners.

Recent research has provided important information about the distribution of the property tax burden across income groups, i.e., progressivity. Conventional views suggest that property taxes may be progressive because of incentives such as homestead property tax exemptions that favor smaller homes over larger homes (e.g., \cite{zodrow14} and \cite{mcmillen20}).  However, recent theoretical and empirical studies show that property taxes tend to be regressive due to wide-spread over assessment of low-valued homes (\cite{mcmillen20} and \cite{berry21}), taxation of home value rather than home equity, and a lack of taxation of other forms of wealth that higher income households are likely to own (\cite{levinson21}).







\section{Theoretical Model}
\subsection{Owners and Renters}
In this section, I develop a dynamic model of a profit maximizing firm that buys and sells units of housing in an asset market and a utility maximizing household that rents a unit of housing in a rental market. Each unit of housing is a bundle of attributes, $z$, and a separate attribute, $q$, of primary interest which one might imagine is an environmental amenity such as clean air. I assume that the asset and rental markets are each in a competitive equilibrium, that the discount rate equals the interest rate, and that participants have complete information, aside from any unexpected shocks.

My approach follows the standard hedonic approach (e.g., \cite{rosen74}) to consumer decision making, but rather than being motivated by the price or user cost of homeownership, the representative consumer or household in my model decides where to rent. I assume that when deciding where to live each time period, a household is maximizing a utility function constrained by their budget and conditional on prices and income. Competing with similar demanders of housing over a set of houses with continuously variable attributes leads to an equilibrium market outcome where each household's marginal willingness to pay for an attribute is equal to the implicit rental price of that housing attribute. I also assume the housing supply is fixed or perfectly inelastic.

At the start of each time period, $t$, the firm purchases a house for $P_t$ which is borrowed at interest rate $r$. After purchasing the house, the firm rents out the unit of housing to a household for $R_t$ and pays costs $C_t$. Later, at $t+1$, the firm sells the house for $P_{t+1}$ and repays the loan plus interest, $(1 + r)P_t$.  The present value of the firm’s profit for a single time period will be,
\begin{equation}
\pi_t = (R_t-C_t)+\left(\frac{P_{t+1}}{1+r}-\frac{P_t(1+r)}{1+r}\right).\label{pi1}
\end{equation}

Where the first term is the net revenue from rent and costs for the time period and the second term
represents the present value (i.e., at the start of the time period, $t$) of house price returns.

Over the same time period, a household rents the house for $R_t$ and has a willingness to pay for housing, $\mathit{WTP}_t=\mathit{WTP}(q_t,z_t,y_t)$ for that same housing unit, where $y_t$ is the renter's income. For simplicity in the mathematical exposition, I assume that consumers have identical WTP functions conditional on income and housing attributes. I assume that there is an equilibrium rent function, $R_t=R(q_t,z_t,P_t,C_t,\bar{y}_t)$, that depends on characteristics of the house and the average income of the population of renters each time period, $\bar{y}_t$. One might estimate this function using the hedonic price method and time series rent data rather than housing price data, though as I will show this is not necessary because house prices can be used to quantify the welfare impacts of a policy or treatment that fundamentally affects the consumers in the rental market. 

Each time period, the consumer’s monetized benefit is:
\begin{equation*}
	 w_t=\mathit{WTP}_t-R_t.
\end{equation*}
Optimization leads to a first order condition that reveals each household's marginal willingness
to pay for housing in time period $t$ at a single point on the household’s demand curve where it is adjacent to the rent function,
\begin{equation}
	\frac{\partial \mathit{WTP}_t}{\partial q_t}=\frac{\partial R_t}{\partial q_t} \label{foc}.
\end{equation}
For the moment, I ignore the relationship between average income, willinginess to pay rent, and equilibrium rent, returning to that question when I consider home ownership below. Returning to the firm, if I assume the market uses information to price houses efficiently, arbitrage will cause today’s price to equal tomorrow’s price, $P_t=P_{t+1}$. Future work might consider relaxing this assumption by incorporating the impacts of uncertainty in the local and macro-economy or by considering amenity levels that vary over time. Now I rewrite the profit equation in \ref{pi1} as,
\begin{equation}
	\pi_t=(R_t-C_t)-P_t*\frac{r}{1+r}.\label{pi1.1}
\end{equation}
Next, utilizing the arbitrage condition, $\pi_t=0$, I solve for the equilibrium rent,
\begin{equation*}
	R_t=C_t+P_t*\frac{r}{1+r}.
\end{equation*}
Which says that over a time period rent is equal to costs during the time period plus the present value of
the interest payment on the cost of the house as capital. If rent were lower than this value, profits
would be negative, firms would exit, house prices would fall, and the interest payments would decline until profits were no longer negative.

For each housing unit I characterize the total welfare, $W_t$ associated with a firm owning and a household renting for a time period,
\begin{eqnarray*}
	W_t & = & w_t+\pi_t,\\
	& = & \mathit{WTP}_t-R_t+0.
\end{eqnarray*}
\subsection{Assessing a Shock}
Now consider a shock, such as an unexpected treatment, that improves environmental quality from $q$ to $q^\prime$ that happens at the end of $t=0$ causing $q$ to increase uniformly in future time periods such that $\Delta q_t=q^\prime_t-q_t>0$ for $t>0$.
The post-shock willingness to pay for the higher amenity level,$\mathit{WTP}^\prime_t=\mathit{WTP}(q^\prime_t,z_t,y)$ along with the post-shock equilibrium rent, $R^\prime_t=R(q^\prime_t,z_t,P^\prime_t,C^\prime_t)$ allows me to write the post-shock welfare measure,
\begin{equation*}
	w^\prime_t=\mathit{WTP}^\prime_t-R^\prime_t \text{ for } t>0
\end{equation*}
The difference between the treated and untreated states for the consumer is,
\begin{eqnarray*}
	\Delta w_t&=&w^\prime_t-w_t=\mathit{WTP}^\prime_t-\mathit{WTP}_t-(R^\prime_t-R_t)\\
	&=&\Delta\mathit{WTP}_t-\Delta R_t
\end{eqnarray*}
%For the producer’s alternative profit in the treated state of reality I write,
%\begin{equation*}
%	\pi^\prime_t = (R^\prime_t-C^\prime_t)+\left(\frac{P^\prime_{t+1}}{1+r}-\frac{P^\prime_t(1+r)}{1+r}\right)\text{ for } t>0.
%\end{equation*}
%which can be simplified in the manner of \ref{pi1.1} by assuming competition and arbitrage drive future profits to zero and markets function efficiently,
%\begin{equation*}
%\pi^\prime_t = (R^\prime_t-C^\prime_t)-P^\prime_t*\frac{r}{1+r}\text{ for } t>0.\label{pi2}
%\end{equation*}

The difference between the treated and untreated states for the producer is zero in all subsequent periods due to the effects of competition and arbitrage, but in the period where the treatment occurs profits are non-zero,
\begin{equation}
	\Delta\pi_0=\frac{P^\prime_1-P_0}{1+r}=\frac{\Delta P_1}{1+r}.\label{pitzero}
\end{equation}
In the context of competitive markets for conventional goods like corn, the existence of potential profits from enhanced production would attract competition, increasing supply, leading to a lower equilibrium price of corn along with more production, more consumption, and more consumer surplus. However due to zoning and fundamental constraints to free entry in land markets, increased willingness to pay leads to higher rents that are capitalized into the price of the real estate. 

Now I calculate the present value of the change in total welfare due to the treatment,
\begin{eqnarray}
	\Delta W_{\mathrm{PV}}&=&\Delta \pi_0+\sum_{t>0}\Delta w_t(1+r)^{-t}\nonumber\\
	&=&\frac{\Delta P_1}{1+r}+\sum_{t>0} (\Delta \mathit{WTP}_t-\Delta R_t)(1+r)^{-t} \label{deltaW1}\\
	&=&\Delta P_{1_{\mathrm{PV}}}+\Delta\mathit{WTP}_\mathrm{PV}-\Delta R_{\mathrm{PV}}.\label{deltaW2}
\end{eqnarray}
This is an intriguing result that helps refine how one might compute the distribution of benefits from public projects that enhance amenities. The house price return is an unexpected profit enjoyed by the owner; this benefit to the firm is immediate and has a clear value, while the sign on the individual differences in the sum of future consumer benefits is indeterminate. The result in \ref{deltaW2} result may seem inconsistent with the usual measure of welfare change, such as in \cite{freeman14} and \cite{banzhaf20}, but next I consider the pattern of changing rent that accompanies the change in price to derive a more familiar measure. 

While $\Delta\pi_t=0$ for $t>0$, it is still helpful to consider the equation for the change in profits to see how rents will change,
\begin{eqnarray}
	\Delta \pi_t &=& \Delta R_t-\Delta C_t-\Delta P_t\left(\frac{r}{1+r}\right)=0 \text{ for } t>0\nonumber\\
	\Delta R_t &=& \Delta C_t+\Delta P_t \left(\frac{r}{1+r}\right)  \text{ for } t>0\label{deltarent}
\end{eqnarray}
Returning to the measure of welfare change in equation \ref{deltaW1}, I substitute for $\Delta R_t$  from \ref{deltarent} and rearrange,
\begin{eqnarray}
	\Delta W_{\mathrm{PV}}&=&\frac{\Delta P_1}{1+r}+\sum_{t>0} \left(\Delta \mathit{WTP}_t-\left(\Delta C_t+\Delta P_t \left(\frac{r}{1+r}\right)\right)\right)(1+r)^{-t} \nonumber\\
	%&=&\frac{\Delta P_1}{1+r}+\sum_{t>0} (\Delta \mathit{WTP}_t-\Delta C_t)(1+r)^{-t}-\Delta P_1 \sum_{t>0}  \left(\frac{r}{1+r}\right)(1+r)^{-t}\nonumber\\
	&=&\frac{\Delta P_1}{1+r}+\sum_{t>0} (\Delta \mathit{WTP}_t-\Delta C_t)(1+r)^{-t}-\frac{\Delta P_1}{1+r}\nonumber\\
	&=&\Delta \mathit{WTP}_{\mathrm{PV}}-\Delta C_{\mathrm{PV}}\nonumber %\label{deltaW2}
\end{eqnarray}
Which is the dynamic counterpart to the usual welfare measure (e.g., \cite{freeman14}). It is important to note that this measure does not convey the imbalanced flow of benefits to the owner and the renter. Indeed, because the public policy that improved the amenity was targeting consumers who would directly benefit, it is a pecuniary externality that competition among renters allows the homeowner to capture a large share of the benefits through rising rent that propels a rising asset value.

\subsection{Marginal Effects}
In the usual approach that is static and does not distinguish between owner and renter, consumer optimization leads to a first order condition,$\frac{\partial{\mathit{WTP}}}{\partial q}=\frac{\partial P_h}{\partial q}$, which is the theoretical result that connects observable prices to WTP. I take a similar approach using the dynamic model described above, but with an infinitesimally small treatment, $\partial q_{t>0}$ applied to all future time periods. I rely on the first order condition in equation \ref{foc} from the renter's optimal choice to reveal information about the consumer's marginal willingness to pay for $q$,

\begin{eqnarray}
\frac{\partial \mathit{WTP}_{\mathrm{PV}}}{\partial q_{t>0}}&=&\sum_{t>0} \frac{\partial \mathit{WTP}_t}{\partial q_t}(1+r)^{-t}\label{pvmwtp}\nonumber\\ 
&=&\sum_{t>0} \frac{\partial R_t}{\partial q_t}(1+r)^{-t}\nonumber\\
&=&\sum_{t>0} \frac{\partial C_t}{\partial q_t}(1+r)^{-t}+\frac{r}{1+r}\sum_{t>0} \frac{\partial P_t}{\partial q_t}(1+r)^{-t}\nonumber
\end{eqnarray}
Which can be rearranged and solved for the partial derivative of the price function,
\begin{equation}
	\frac{\partial P_{1_{\mathrm{PV}}}} {\partial q_{t>0}}=\frac{\partial \mathit{WTP}_{\mathrm{PV}}}{\partial q_{t>0}}-\frac{\partial C_{\mathrm{PV}}}{\partial q_{t>0}} \label{marginalPrice}.
\end{equation}
%Where I have relied on the following relations,
%\begin{eqnarray*}
%\frac{r}{1+r}\sum_{t>0} \frac{\partial P_t}{\partial q_t}(1+r)^{-t}&=&\frac{r}{1+r}\sum_{t>0} (\frac{\partial R_t}{\partial q_t}-\frac{\partial C_t}{\partial q_t})(1+r)^{-t}\\
%&=& \frac{1}{1+r}*\frac{\partial P_1}{\partial q_{t>0}}\\
%&=& \frac{\partial P_{1_{\mathrm{PV}}}} {\partial q_{t>0}}
%\end{eqnarray*}
%\begin{eqnarray}
%	P_0=\sum_{t>0}(R_t-C_t)(1+r)^{-t}\\
%	P_{1_{\mathrm{PV}}}=\frac{P_1}{1+r}=\frac{P_0}{1+r}\\
%	\frac{\partial R_t}{\partial q_t}-\frac{\partial C_t}{\partial q_t}=\frac{\partial P_t}{\partial q_t}*\frac{r}{1+r}
%\end{eqnarray}
If I assume that costs do not change, then \ref{marginalPrice} is a clear dynamic, present-value counterpart to the standard tangency condition used in the static ownership literature. If marginal costs are positive (e.g., increased property tax liabilities or increased maintenance costs due to satisfy a different group of renters), then the implicit price tends to underestimate marginal WTP. On the other hand, if marginal costs are negative (e.g., reduced defensive expenditures such as air purifiers), then the implicit price tends to overestimate marginal WTP.

\subsection{Renters Who Own}
Next, considering renters who also own a home (though not necessarily the one they rent), I evaluate the effects on the market of increased wealth due to a shock (e.g., an unexpected policy or treatment) that increases profits. In this section, I use $\rho$ and $o$ superscripts to indicate the housing asset the individual rents or owns, respectively. If I assume that increases in wealth due to profits from homeownership translate to smooth increases in consumption in perpetuity (implying the household's marginal propensity to consume (MPC) the perpetuity payment equals one), then consumption becomes a function of attributes of the rented and owned houses, such as $q^\rho$ and $q^o$. So I define WTP for a renter who also own a house,
%\begin{eqnarray*}	
%	\mathit{WTP}_t^{\mathit{owner}}&=&\mathit{WTP}(q_t,z_t,y_t(q_t^o)) \text{ and} \\
%	\mathit{WTP'}_t^{\mathit{owner}}&=&\mathit{WTP}(q'_t,z_t,y_t(q'_t^o)) \text{ for } t>0\\
%	&=&	\mathit{WTP}(q'_t,z_t,y_t+\pi_0*r) \text{ for } t>0.
%\end{eqnarray*}
\begin{eqnarray*}
	\mathit{WTP}_t^{\prime\mathit{owner}}&=&\mathit{WTP}(q_t^{\prime \rho},z_t^\rho,y_t^\prime)\\
	&=&\mathit{WTP}(q_t^{\prime \rho},z_t^\rho,y_t+r*\pi^o_{\mathrm{PV}})
	%&=&\mathit{WTP}(q_t^{\prime \rho},z_t^\rho,y_t+\Delta \pi_0^o*r) \text{ for } t>0
\end{eqnarray*}
Where $r*\pi_{\mathrm{PV}}^o$ is the perpetual income payment from any profits associated with homeownership. Implicitly, the previous WTP measures are for renters who do not own a house.

Now I consider the marginal welfare measure from an increase in the amenity at all relevant houses for a market that includes renter-owners. I start by defining a idiosyncratic and systematic component for each house's amenity, such that $q_{t}^i=\eta_{t}^i+\epsilon_t$, where the superscript, $i$, indexes the houses in a region. Next I consider a small shock to future levels of the systematic component of the amenity, $\epsilon_t$,% such that $\epsilon^\prime_t-\epsilon_{t}>0$ for $t>0$.  

\begin{eqnarray}
	\frac{\partial \mathit{WTP}^{\mathit{owner}}_{\mathrm{PV}}}{\partial \epsilon_{t>0}}&=&\sum_{t>0}\frac{\partial \mathit{WTP}_t}{\partial q^\rho_t}*(1+r)^{-t} \nonumber\\
	&& +\sum_{t>0}\frac{\partial \mathit{WTP}_t}{\partial y_t}*\frac{\partial y_t}{\partial q_t^o}(1+r)^{-t} \nonumber\\
	%&=&\sum_{t>0}\frac{\partial \mathit{WTP}_t^{\mathit{renter}}}{\partial q_t}*(1+r)^{-t}\nonumber\\
	%&& +\sum_{t>0}\frac{\partial \mathit{WTP}_t^{\mathit{renter}}}{\partial y_t}*\frac{\partial y_t}{\partial q_t}(1+r)^{-t} \nonumber
	&=&\sum_{t>0}\frac{\partial \mathit{WTP}_t}{\partial q_t^\rho}*(1+r)^{-t}\nonumber\\
	&& +\sum_{t>0}\frac{\partial \mathit{WTP}_t}{\partial y_t}*r*\frac{\partial P^o_{1_\mathrm{PV}}}{\partial q_t^o}(1+r)^{-t} \label{mwtpOwner0}
\end{eqnarray}
Where $P^o_{1_\mathrm{PV}}$ is the present value of the house the individual owns at the start of $t=1$. Next I assume that the number of renter-owners is small, so their actions do not lead to indirect effects  elsewhere in the housing market such as through increased average income. I also assume $\partial \mathit{WTP}_t/\partial y_t=\lambda$, where $\lambda$ is constant over time, allowing me to write,
\begin{eqnarray}
	\frac{\partial \mathit{WTP}^{\mathit{owner}}_{\mathrm{PV}}}{\partial \epsilon_{t>0}} 
	&=&\frac{\partial \mathit{WTP_{\mathrm{PV}}}}{\partial q^\rho_{t>0}} + \lambda*\frac{\partial P^o_{1_\mathrm{PV}}}{\partial q_{t>0}^o} \label{mwtpOwner1}.
\end{eqnarray}
The first term in the right hand side of \ref{mwtpOwner1} describes the renter's future optimizing decisions in the rental market, where choosing from housing options causes $q_t^\rho$ to vary, so I solve equation \ref{marginalPrice} for marginal WTP and substitute it into the first term of \ref{mwtpOwner1}. Then I assume that the derivatives of the cost and price functions with respect to the amenity are the same for the house that the individual rents and the house they own, abstracting away from the distinction between the two houses, 

\begin{equation}
	\frac{\partial \mathit{WTP}^{\mathit{owner}}_{\mathrm{PV}}}{\partial \epsilon_{t>0}} =	\frac{\partial C_{\mathrm{PV}}}{\partial q_{t>0}}+(1+\lambda)*\frac{\partial P_{1_\mathrm{PV}}}{\partial q_{t>0}}.
\end{equation}

Rearranging the result highlights the relationship between marginal willingness to pay and marginal sale price for permanent changes in an amenity,
\begin{equation}
	\frac{\partial P_{1_{\mathrm{PV}}}} {\partial q_{t>0}}=\left(
	\frac{\partial \mathit{WTP}^{\mathit{owner}}_{\mathrm{PV}}}{\partial q_{t>0}}-\frac{\partial C_{\mathrm{PV}}}{\partial q_{t>0}}\right)\left(\frac{1}{1+\lambda}\right) \label{marginalPriceOwner}
\end{equation}

In the narrow case of housing markets where few renters own, the marginal willingness to pay of renter-owners is underestimated by the implicit price from the hedonic price function due to the exclusion of income effects. I leave for future work modeling of more complicated interactions among many renters who own.

%Conversely, we expect that in markets with a substantial number of renter-owners, there will be a change in average income due to the increased amenity values of the houses. We can include the effect of resorting and bidding up of the rent function and rewrite the renter's first order condition,
%
%\begin{equation}
%\frac{\partial \mathit{WTP}_t}{\partial q_t}=\frac{\partial R_t}{\partial q_t} + \frac{\partial R_t}{\partial \bar{y}_t}*\frac{\partial \bar{y}_t}{\partial q_t}\\ \label{focRenter}
%\end{equation} 
%At the same time, the renter-owner's optimal equilibrium behavior would be described by a first order condition including the income effects that would be generated during the sorting implied  by equilibrium,
%%\frac{\partial \mathit{WTP_{\mathrm{PV}}}}{\partial q^\rho_{t>0}} + \lambda*\frac{\partial P^o_{1_\mathrm{PV}}}{\partial q_{t>0}^o} \label{mwtpOwner1}.
%\begin{equation}
%\frac{\partial \mathit{WTP}_t}{\partial q_t}+\lambda*r*\frac{\partial P_{1_{\mathrm{PV}}}}{\partial q_{t>0}}=\frac{\partial R_t}{\partial q_t} + \frac{\partial R_t}{\partial \bar{y}_t}*\frac{\partial \bar{y}_t}{\partial q_t}\\ \label{focOwner}
%\end{equation} 
%
%Next, we assume that increases in WTP from increases in average income translate linearly into increases in rent proportional to willingness to pay, 
%\begin{eqnarray}
%\frac{\partial R_t}{\partial \bar{y}_t}=\alpha*\frac{\partial \mathit{WTP}_t}{\partial y_t}=\alpha*\lambda
%\end{eqnarray}
%Next, we assume a constant share of households are renter-owners, $\gamma$, allowing me to write,
%\begin{equation}
%	\frac{\partial\bar{y}_t}{\partial q_t}=\gamma*r*\frac{\partial P_{1_{\mathrm{PV}}}} {\partial q_{t>0}}
%\end{equation}
%
%We then use substitute into \ref{focOwner}, which is then substituted into \ref{mwtpOwner0}, again abstracting from any difference in the house owned and rented by a renter-owner and solving for marginal price,
%\begin{equation}
%\frac{\partial P_{1_{\mathrm{PV}}}} {\partial q_{t>0}}=\left(
%\frac{\partial \mathit{WTP}^{\mathit{owner}}_{\mathrm{PV}}}{\partial q_{t>0}}-\frac{\partial C_{\mathrm{PV}}}{\partial q_{t>0}}\right)\left(\frac{1}{1+\lambda*\left( 1 + \alpha * \gamma\right)}\right) \label{marginalPriceManyOwners}
%\end{equation}



\subsection{Taxes}
It can be useful to separate taxes out as a component of costs to consider how changes in property tax rates impact renters and owners. Assume firms pay a property tax at the end of each time period on $P_{t}$, such that the value of the tax at time period $t$ is $T_t=g*P_{t}*\frac{1}{1+r}$. I incorporate this into the firm's profit equation from \ref{pi1},
\begin{equation*}
	\pi^T_t = (R^T_t-C^T_t)+\left(\frac{P^T_{t+1}}{1+r}-\frac{P^T_t(1+r+g)}{1+r}\right).\label{pi1T}
\end{equation*}
The tax is imposed at the end of $t=0$. Markets subsequently adjust to a new equilibrium with competition returning profits to zero and allowing me to express equilibrium rents as a function of the new tax,
\begin{equation*}
	R^T_t=C^T_t-\left(\frac{P^T_{t+1}}{1+r}-\frac{P^T_t(1+r+g)}{1+r}\right) \text{ for }t>0
\end{equation*}
Then efficient use of information and competition cause prices in the next period to equal prices in this period, allowing for simplification of the rental equation,
\begin{equation*}
R^T_t=C^T_t-\frac{P^T_{t}(g+r)}{1+r}\text{ for }t>0.
\end{equation*}
Because the tax does not influence decisions made at the margin and the housing supply is fixed, I assume rents and costs do not change due to the property tax. 


The firm's profits are driven by competition to zero in subsequent years, but in year 0 the firm's profits will be non-zero due to the adjustment in price after $t=0$ that occurs because of the tax,
\begin{equation*}
	\pi^T_0 = (R^T_0-C^T_0)+\left(\frac{P^T_{1}(1-g)}{1+r}-P_0\right).\label{piT0}
\end{equation*}
Because of my assumption that rents and costs do not change due to the property tax, I solve for the change in price due to the tax,
\begin{eqnarray}
	\Delta^T R_t&=&\Delta^T C_t+P^T_{t}\frac{g+r}{1+r}-P_{t} \frac{r}{1+r},\nonumber \\
	0&=&0+P^T_{t}\frac{g+r}{1+r}-P_{t} \frac{r}{1+r},\nonumber \\
	P^T_t&=& P_t\frac{r}{g+r}.\label{TxPrice} \label{Ptax}
\end{eqnarray}
Where $\Delta^T$ denotes the difference between the state of reality with the tax and without. As long as g and r are positive, the property tax will cause the price of the house to fall to a new equilibrium level.
Next I consider the change in profits during $t=0$ as a function of $g$ and $P_0$,
\begin{eqnarray*}
	\Delta^T \pi_0 &=& P_1^T\frac{1-g}{1+r}-P_1\frac{1}{1+r}\\
	&=& P_1\left(\frac{r(1-g)}{(g+r)(1+r)}-\frac{1}{1+r}\right)\\
	&=&-P_0\frac{g}{g+r}.
\end{eqnarray*}
Where I have relied on equality of prices between the two scenarios in $t=0$ and also that without a tax $P_t=P_0$. The total welfare impact on the firm and household is simply this change in profits in time period zero.
Given that rents do not change but the interest payments on the price of the house decrease, I decompose the change in rents due to the tax,
\begin{eqnarray*}
	\Delta^T R_t&=&P^T_t\frac{g+r}{1+r}-P_t\frac{r}{1+r}\\
	0&=&P^T_t\frac{g}{1+r}-(P_t-P_t^T)\frac{r}{1+r}.
\end{eqnarray*}
The first term is the present value of the future annual tax payments and the second term is the present value of future reductions in rent due to a lower house price. The two terms exactly offset each other. Notably, relaxing my assumption of a fixed or perfectly inelastic housing stock would allow for a reduction in investment in housing due to lower returns, offsetting to some extent the reduction in the private price of the house (i.e., incomplete capitalization of the property tax), leading to an increase in rent.

My analysis of welfare effects above do not consider the possibility that taxes already exist. Because property taxes distort the information about the occupant's preferences, it is important to correct this effect when quantifying benefits. I return to \ref{marginalPrice} and substitute for $P_1$ using \ref{Ptax} to find the welfare effect when observed prices include complete capitalization of existing property taxes,
\begin{equation}
\frac{\partial P^T_{1_{\mathrm{PV}}}} {\partial q_{t>0}}=\frac{r}{g+r}*\left(\frac{\partial \mathit{WTP}_{\mathrm{PV}}}{\partial q_{t>0}}-\frac{\partial C_{\mathrm{PV}}}{\partial q_{t>0}}\right). \label{marginalPriceT}
\end{equation}
This equation indicates that in the presence of existing property taxes, marginal implicit prices should be scaled up by the ratio $\frac{g+r}{r}$ when measuring net marginal benefits because price changes only measure the private benefit of the increased rents paid by renters. This result is identical to equation 4 in \cite{niskanen77}, but my approach emphasizes the source of these values is derived from a home occupant's willingness to pay rent, while their interest is focused on including in welfare assessments the increased value of property taxes that comprise the government's share of a property's value.

Similarly, I compute the necessary adjustment for owner-occupied housing where the owner experiences income effects by substituting \ref{Ptax} into \ref{marginalPriceOwner},
\begin{equation}
\frac{\partial P^T_{1_{\mathrm{PV}}}} {\partial q_{t>0}}=\frac{r}{(1+\lambda)(g+r)}*\left(
\frac{\partial \mathit{WTP}^{\mathit{owner}}_{\mathrm{PV}}}{\partial q_{t>0}}-\frac{\partial C_{\mathrm{PV}}}{\partial q_{t>0}}\right) \label{marginalPriceOwnerT}
\end{equation}
Which illustrates that for renter-owners, the existence of property taxes and income effects further divides implicit price from net marginal benefits.

\subsection{A Corrective Tax}
While pecuniary externalities are not an inefficiency that merits correction on the grounds of economic efficiency, political and ethical perspectives may call for such a correction. For example, a public expenditure that enhances air quality is likely designed to benefit the households in a region, regardless of whether or not they own a house. But as shown above, homeowners can charge for access to the clean air with the increased rents that future tenants are willing to pay. Consequently, I am interested in identifying a property tax on owners and subsidy to renters that can be designed to exactly offset the pecuniary externality that transfers a portion of the benefits of the amenity change from renter to owner. 

Consider an increase in amenities that causes a change in welfare, as described in \ref{deltaW2}. Next, let $T_t$ for $t>0$ be a tax on the home owner, and let $S_t$ for $t>0$, be an equivalent subsidy paid to the renter. I incorporate these into the welfare measure in \ref{deltaW2},
\begin{eqnarray*}
	\Delta W_{\mathrm{PV}}&=& \left(\Delta P_{\mathrm{PV}}-T_{\mathrm{PV}}\right)+\left(\Delta\mathit{WTP}_\mathrm{PV}-\Delta R_{\mathrm{PV}}+S_{\mathrm{PV}}\right).\nonumber
\end{eqnarray*}
To offset the owner's unexpected profits due to the increased amenities in \ref{pitzero}, I set the tax equivalent to the change in profits that come about due to the change in amenities,
\begin{eqnarray*}
 \Delta \pi_{\mathrm{PV}}&=&T_{\mathrm{PV}}\\
\frac{\Delta P_1}{1+r}&=&P_1\frac{g}{g+r}
\end{eqnarray*}
which can be solved for the corrective property tax rate, $g$,
\begin{equation}
	g=\frac{\% \Delta P_1*r}{1+r-\% \Delta P_1} \label{gcorrect}
\end{equation}
Because hedonic models are often estimated with a log-transformed dependent variable, estimating \ref{gcorrect} is straightforward, particularly for small changes in $q_{t>0}$ when regression coefficients approximate measures of percentage change. 

To avoid the situation where the firm captures the benefits of the subsidy, the payment of the subsidy would need to be independent of the renter's decision about where to live. Renter-owners would experience no price change and they would receive a subsidy as well as pay a tax. Firms that own multiple houses would pay the tax multiple times while households receive all of the subsidies. Firms would experience no change in profits because the pecuniary impact of the corrective tax would cancel out the pecuniary impact of the amenity improvement.
Notably, this policy is revenue neutral for the government and the firm because, 
\begin{equation*}
	P_0*\frac{g}{g+r}=T_{\mathrm{PV}}=S_{\mathrm{PV}}=\Delta \pi_{\mathrm{PV}}=-\Delta^T \pi_{\mathrm{PV}}=\sum_{t>0}\frac{g*P_t^T}{(1+r)^{-t}}.
\end{equation*}
Where the first term is the change in profits from the tax, the second term is the tax, the third is the subsidy, the fourth is the benefit to the firm of the amenity improvement, the fifth is the property tax burden for the firm, and the last term is the government's tax revenue. 

 
\section{Empirical Application}
In this section, I apply the theoretical model to a large, often-cited study by \cite{chaygreenstone05} that estimates the benefits of the Clean Air Act Ammendments. While this application is coarse, it is useful for illustrating the magnitude of costs to renters due to the pecuniary externalities that allow privatization of public goods.  Because logarithmic transformations of the dependent variable, the house price, are standard in empirical applications such as \cite{chaygreenstone05}, it is straightforward to estimate $g$ and related measures for a known value of $\%\Delta P_1 $. 

\cite{chaygreenstone05} quantified the benefits to households of increased air quality, using non-attainment status under the Clean Air Act Amendments as an exogenous instrument for quantifying the causal impact of changes in total suspended particulate (TSP) concentrations on housing values in the non-attaining counties facing new incentives to enhance air quality. The authors find that for these non-attaining counties the Clean Air Act Amendments caused TSP concentrations to decrease by about 10 $\mu g/m^3$ during the 1970s. The authors apply a variety of econometric techniques for estimating hedonic models to infer that housing prices increase by about 0.28 percent for each $\mu g/m^3$ decrease in TSP concentrations. Then they assume constant marginal WTP and infer that the improved air quality of 10 $\mu g/m^3$ increased housing values in non-attaining counties during the 1970s by \$45 billion (in 2001 dollars) or 2.8\%. \cite{chaygreenstone05} go on to discuss how the resulting value represents both an estimate of willingness to pay and an estimate of the increased price of the houses. I would add that this value estimates the present value of the increase in future rental prices that consumers will pay (implicitly for owner-occupants) for cleaner air. Using this value as an estimate of WTP also implies that renters enjoyed no welfare gains as they paid for increasingly clean air.

\cite{chaygreenstone05} provide helpful summary statistics including mean \% owner occupied across the 988 counties in their sample, which fell from about 68\% to 62\% in the non-attainment counties during the 1970s. Assuming that the renter-occupied housing values respond to particulate concentrations in the same manner as the owner-occupied housing used in their regressions, this implies that about 35\% of the \$45 billion (i.e., about \$24 Billion in 2021 dollars) in increased housing values come from increased present and future rental payments from renters to their landlords. 
%according to https://www.bls.gov/data/inflation_calculator.htm, the cpi for 7/2001 to 7/2021 is 1.54

Using \ref{gcorrect} I estimate a corrective property tax rate of $g\approx 0.137\%$ at a 5\% interest rate and $g\approx 0.084\%$ at a 3\% interest rate. Applied to the mean housing value of \$86,900 (\$134,000 in 2021 dollars), this represents an average annual property tax increase in 2021 dollars of \$184 at a 5\% interest rate or \$112 at a 3\% interest rate, which applied to the 19 million homes in the unattaining counties, translates to annual tax revenue of \$3.48 billion %(.028*.05)/(1+.05-.028)*86900*1.54*19
or \$2.13 billion%(.028*.03)/(1+.03-.028)*86900*1.54*19
, respectively. The average property tax increase would also be the average annual subsidy per household that the tax could fund. 

Using the result in \ref{marginalPriceT}, I also correct willingness to pay estimates for existing property taxes that reduce the private sale price of the house relative to the marginal benefits by the ratio $\frac{r}{g+r}$. To estimate the present value of the change in willingness to pay in the presence of existing property taxes, I assume a nominal property tax rate of $g=1.5\%$, and at $r=5\%$ or $r=3\%$, the private sale price should be scaled up by the factors $1.3$ or $1.5$, respectively. This suggests that the aggregate benefits of the air quality enhancements considered by \cite{chaygreenstone05} are substantially higher than the value of the privately owned share of each house indicates. From renters, the increase in present and future rent payments due to the amenity increase transfer an additional \$7.2 billion or \$12 billion, at $r=5\%$ and $r=3\%$ respectively, to the local and state governments. The corrective tax calculation is unaffected, because the reduction in the private price prevents the increase in property taxes and rent.
%average property tax rate inspired by https://www2.census.gov/library/publications/2010/compendia/statab/130ed/tables/11s0446.pdf

%**Additionally**, quantification of the additional marginal benefits experienced by renter-owners due to the income effects requires relatively little data if an existing hedonic price study estimates a coefficient for the income of the occupant that can be used as an estimate of $\lambda$.



\section{Discussion}
Throughout this paper I have mostly ignored the econometric challenges associated with quantifying changes in WTP that dominate the literature. It seems likely that the manner in which future willingness to pay, future rents, future costs, etc. are capitalized into today's prices increase the challenge associated with identifying an individual's demand for housing. Efficient markets form expectations about future values, so welfare estimates derived from the asset market provides something of a market consensus about how future conditions and future preferences of future occupants will translate into the relevant measures of future supply and demand in rental markets. Ultimately, I expect that the enhanced clarity that comes from separating the renter from the owner will  determine the appropriate techniques for fully quantifying the distribution of welfare changes.

The scope of a property owner's legal rights is an important component of evaluating the feasibility of applying a corrective tax like I describe above. Societal and Political views on increased taxes suggest there would be substantial opposition to new property taxes. However, people are accustomed to paying for government provided services with property taxes in a pattern that \cite{tiebout56} famously recognized as a form of economically efficient market activity. As changes in populations and ecosystems lead to increased provisioning and management of ecosystem services by local, state, and federal governments, it is natural to reconsider a property owner's rights to freely enjoy profits derived from public goods.

The non-zero probability of moving that renters face each time period is built into consumer decision making and is thus reflected by housing prices. The expectations of firms and households about this process are built into observed prices. My simple model omits this information, which I expect would lead to underestimation of benefits due to underestimated costs such as moving costs. It would be interesting to extend my model to include forward looking renters (e.g., \cite{bishop19}) to better understand the impacts of frequent moving on renters and to assess the comparative benefit to owners.  

The effects on house prices and owner well-being identified in this paper suggest an import consideration in the development of sorting models such as the single cross (\cite{banzhaf20}). Enhanced amenity values in a locality may lead to re-sorting in two ways. First if some houses become relatively more valuable compared to others, a re-ordering of houses. Simultaneously, the sorting order of the households will change if some of them are owners too, as their willingness to pay may rise enough to change the ordering of households. Furthermore, if higher income households tend to live in areas with the highest concentration of homeownership, multiple sub-populations of homeowners and homes may arise without any underlying difference in preferences. Alternatively, if households have already segregated based on ownership, sorting may be mature and household ordering may be stable.

A key assumption behind my use of the term, pecuniary externality, is that the policy in consideration is targeting households not property owners. While property taxes are familiar, subsidies targeting renters are uncommon, likely because they would effectively be an amenity that would be captured by the owner. A lump sum transfer to all resident in the affected region of the total tax receipt divided across all houses would be one approach to paying out the subsidy to renters. Renters in low-rent housing, who are likely to have a high marginal utility of income and a high marginal propensity to consume would also receive a larger subsidy than their landlords would pay in property taxes, a potential path toward restorative justice. 

In the context of the "just green enough" \cite{curran12} approach to avoiding environmental gentrification, the taxation and subsidy approach described here does not address the impacts of environmental enhancements on industry, so working class households may be negatively affected through reduced employment opportunities and increased commuting costs. Alternatively, subsidies may facilitate access to education and other government services that increases opportunities for families, offsetting changes in local employment opportunities. An alternative use of tax revenues that would support local industry would be to use tax revenues to compensate local firms for reductions in pollution.

I ended this paper's introduction by referencing the potential pareto improvements sought by policy makers using hedonic price analysis to inform cost-benefit analysis. However, the strong role of a property's future rental price in determining the property's current price combined with well documented environmental gentrification suggests that policymakers must consider the welfare of current residents as well as future residents to identify potential pareto improvements even for policies with a limited geographic scope. Increased benefits to homeowners may be directly tied to subtantial losses for renters who have little choice but to flee rising rents worth more (in the budget-dependent measures of microeconomics) to future residents than themselves.

Future work may quantify how ownership effects induce sorting and segregation in the absence of distinct preferences. Extensions similar to \cite{bishop19} might be used to modify the consumer's first order condition to assess how forward looking consumers respond to varying ammenity levels or moving costs when choosing the optimal home to rent. Explicity spatial models would be useful for better understanding how a corrective tax and subisdy would function when ammenity changes are spatially uneven. Extending the work of \cite{kanemoto88} to distinguish between renter-owners and renters in a dynamic, general equilibrium framework would be another interesting extension of this work. Similarly, the analysis of \cite{gervais2002} might be modified to incorporate income effects and multiplier effects associated with groups of asset buying households who bid up each other's housing prices. Incorporating price inflation and the distinct tax treatments of owner-occupied  houses (e.g., \cite{poterba84}) into this model will help clarify the full benefits and consequences of homeownership (or a lack thereof), providing a useful means for anticipating and offsetting environmental injustice.

An important limitation of this work is the absence of changes in investment that if present, would limit the capitalization of both property taxes and enhanced environmental benefits. Incomplete capitalization of property taxes will tend to favor the property owner who will be able to transfer some of the property tax burden to the renters. Notably, this perspective is natural for my model and in contrast to the claim of \cite{sirmans08} that the burden is transferred to future owners. I expect that incomplete capitalization of both enhanced environmental benefits and increased property taxes would tend to offset each other, so uncertainty about housing supply elasticities may not interfere with the distributional effects or the calculation of the corrective tax. 

Because assessment practices and the structure of the tax code cause lower income households to pay higher effective property tax and wealth rates, investments in low-income housing are already disincentivized relative to higher-income housing. Aside from corrective tax policy, information about how renters are affected by negative pressures on housing supply and how rents and low income households are affected by amenity enhancements may be useful for decision makers and the political process.


\section{Conclusions}
The contributions in this paper provide a foundation for identifying and rectifying past social injustices. In this paper, I demonstrate how the fundamentally monopolistic nature of land gives rise to pecuniary externalities that allow landowners to sell public goods like air quality in a private rental market. For renter occupants, enhancements to public goods increase rent and transfer the bulk of measurable benefits to landowners as perpetually increased rents. Simultaneously, user costs consume these increased rents, but landowners enjoy an instantaneous increase in wealth due to the increased value of their property. The property tax I identify affords society the option of recapturing the increased wealth generated by public goods. I also identify a subsidy that may be useful for offsetting increases in rent that may otherwise harm current residents, causing them to move away. Notably, this suggests that using hedonic price analysis for cost-benefit analysis may lead analysts to erroneously recommend policies that fail to achieve even a potential pareto improvement. Future research is needed to better understand how to design such a subsidy to achieve the desired outcome.

While I have focused on single family homes in the development of my model, these results are applicable to any productive asset, such as a factory. For example, if an industry uses natural resources in production, then public enhancements to natural resources will tend to reduce production costs and to the extent that competition is not perfect, this would increasing annual profits and equity. Equity markets likely benefit tremendously from enhanced environmental quality, suggesting further opportunities for redistributing the private benefits of public goods back to the public.
 


\begin{thebibliography}{}

\bibitem[\protect\citeauthoryear{Aaronson et. al.}{2021}]{aaronson2021}
\textup{Aaronson, D., Faber, J., Hartle, D., Mazumder, B., Sharkey, P.} %(2021):
`` The Long-Run Effects of the 1930s HOLC “Redlining” Maps on Place-Based
Measures of Economic Opportunity and Socioeconomic Success,''
\textit{Regional Science and Urban Economics}, 86 (2021), 103622.
\endbibitem

\bibitem[\protect\citeauthoryear{Bajari et. al.}{2012}]{bajari12}
\textup{Bajari, P., Fruehwirth J.C., Kim, K.I., Timmins, C.} % (2012):
`` A Rational Expectations Approach to Hedonic Price Regressions with Time-Varying Unobserved Product Attributes: The Price of Pollution,''
\textit{American Economic Review}, 102:5 (2012), 1898--1926.
\endbibitem

\bibitem[\protect\citeauthoryear{Banzhaf}{2020}]{banzhaf20}
\textup{Banzhaf, S.} % (2020):
``Panel Data Hedonics Rosen’s First Stage as a “Sufficient
Statistic",''
\textit{International Economic Review}, 61:2 (2020), 973--1000.
\endbibitem

\bibitem[\protect\citeauthoryear{Banzhaf}{2021}]{banzhaf21}
\textup{Banzhaf, H.S.} % (2021):
``Difference-in-Differences Hedonics,''
\textit{Journal of Political Economy}, 129:8 (2021), 2385--2414.
\endbibitem


\bibitem[\protect\citeauthoryear{Bento, Lang, and Freedman}{2015}]{bento15}
\textup{Bento, A., Freedman, M., and Lang, C.} % (2015):
``Who Benefits from Environmental Regulation? Evidence from the Clean Air Act Amendments,''
\textit{Review of Economics and Statistics}, 129:8 (2015), 2385--2414.
\endbibitem

\bibitem[\protect\citeauthoryear{Berry}{2021}]{berry21}
\textup{Berry C. R.} % (2021):
``Reassessing the Property Tax,'' (2021)
Available at SSRN: https://ssrn.com/abstract=3800536 or http://dx.doi.org/10.2139/ssrn.3800536
\endbibitem

\bibitem[\protect\citeauthoryear{Banzhaf, Ma, and Timmins}{2019}]{banzhafJustice19}
\textup{Banzhaf, S., Ma, L. and Timmins, C.} % (2019):
``Environmental Justice: The Economics of Race, Place, and Pollution,''
\textit{Journal of Economic Perspectives}, 33:1 (2019), 185--208.
\endbibitem

\bibitem[\protect\citeauthoryear{Bishop, et al.}{2020}]{bishop20}
\textup{Bishop, K.C., Kuminoff, N.V., Banzhaf, H.S., Boyle, K.J., von Gravenitz, K., Pope, J.C., Smith, V.K. and Timmins, C.D., } % (2020):
``Best Practices for Using Hedonic Property Value Models to Measure Willingness to Pay for Environmental Quality,''
\textit{Review of Environmental Economics and Policy}, 14:2 (2020), 260--281.
\endbibitem

\bibitem[\protect\citeauthoryear{Bishop and Murphy}{2011}]{bishop11}
\textup{Bishop, K.C. and Murphy, A.D.} % (2011):
``Estimating the Willingness to Pay to Avoid Violent Crime: A Dynamic Approach,''
\textit{American Economic Review}, 101:3 (2011), 625--629.
\endbibitem

\bibitem[\protect\citeauthoryear{Bishop and Murphy}{2019}]{bishop19}
\textup{Bishop, K.C., and Murphy, A.D.} % (2019):
``Valuing Time-Varying Attributes Using the Hedonic Model: When is a Dynamic Approach Necessary?,''
\textit{Review of Economics and Statistics}, 101:1 (2019), 134--145.
\endbibitem

\bibitem[\protect\citeauthoryear{Bishop and Timmins}{2018}]{bishoptimmins18}
\textup{Bishop, K.C., and Timmins, C.} % (2018):
``Using Panel Data to Easily Estimate Hedonic Demand Functions,''
\textit{Journal of the Association of Environmental and Resource Economists}, 5:3 (2018), 517--543.
\endbibitem

\bibitem[\protect\citeauthoryear{Buchanan and Stubblebine}{1962}]{buchananstubblebine}
\textup{Buchanan, J.M. and Stubblebine, W.C.} % (1962):
``Externality,''
\textit{Economica}, 29:116 (1962), 371-384.
\endbibitem

\bibitem[\protect\citeauthoryear{Bui and Mayer}{2003}]{bui03}
\textup{Bui, L.T. and Mayer, C.J.} % (2003):
``Regulation and Capitalization of Environmental Amenities: Evidence from the Toxic Release Inventory in Massachusetts,''
\textit{Review of Economics and Statistics}, 85:3 (2003), 693--708.
\endbibitem


\bibitem[\protect\citeauthoryear{Cabral and Hoxby}{2012}]{cabralhoxby}
\textup{Cabral, M., and Hoxby, C.} % (2012):
``The Hated Property Tax: Salience, Tax Rates, and Tax Revolts,''
\textit{NBER Working Paper Series}, Working Paper 18514 (2012).
\endbibitem

\bibitem[\protect\citeauthoryear{Chay and Greenstone}{2005}]{chaygreenstone05}
\textup{Chay, K.C. and Greenstone, M.} % (2005):
``Does Air Quality Matter? Evidence from the Housing Market,''
\textit{The Journal of Political Economy}, 113:2 (2005), 376--424.
\endbibitem

 
\bibitem[\protect\citeauthoryear{Curran and Hamilton}{2012}]{curran12}
\textup{Curran, W. and Hamilton, T.} % (2012):
``Just Green Enough: Contesting Environmental Gentrification in Greenpoint, Brooklyn,''
\textit{Local Environment}, 17:9 (2012), 1027--1042.
\endbibitem

 
\bibitem[\protect\citeauthoryear{Dávila and Korinek}{2018}]{davila18}
\textup{Dávila, E. and Korinek, A.} % (2018):
``Pecuniary Externalities in Economies with Financial Frictions,''
\textit{The Review of Economic Studies}, 85:1 (2018), 352--395.
\endbibitem
 
\bibitem[\protect\citeauthoryear{England}{2016}]{england16}
\textup{England, R. W. } % (2016):
``Tax Incidence and Rental Housing: A Survey and Critique of Research,''
\textit{National Tax Journal}, 69:2 (2016), 435--460.
\endbibitem

\bibitem[\protect\citeauthoryear{Freeman}{1974}]{freeman74}
\textup{Freeman, A.M.} % (1974):
``On Estimating Air Pollution Control Benefits from Land Value Studies,''
\textit{Journal of Environmental Economics and Management}, 1:1 (1974), 74--83.
\endbibitem

\bibitem[\protect\citeauthoryear{Freeman}{1980}]{freeman80}
\textup{Freeman, A.M.} % (1980):
``Land Prices Substantially Underestimate the Value of Environmental Quality: A Comment,''
\textit{The Review of Economics and Statistics}, 62:1 (1980), 154--156.
\endbibitem


\bibitem[\protect\citeauthoryear{Freeman}{1999}]{freeman99}
\textup{Freeman, A.M.} % (1999):
\textit{The Measurement Of Environmental And Resource Values: Theory And Methods}.
Washington DC, U.S.A.: Resources for the Future (1999).
\endbibitem
 
\bibitem[\protect\citeauthoryear{Freeman, Herriges, and Kling}{2014}]{freeman14}
\textup{Freeman III, A.M., Herriges, J.A., and Kling, C.L.} % (2014):
\textit{The Measurement Of Environmental And Resource Values: Theory And Methods}.
Washington DC, U.S.A.: Routledge (2014).
\endbibitem 

\bibitem[\protect\citeauthoryear{Gervais}{2002}]{gervais2002}
\textup{Gervais, M.} % (2002):
``Housing Taxation an Capital Accumulation,''
\textit{Journal of Monetary Economics}, 49 (2002), 1461--1489.
\endbibitem

\bibitem[\protect\citeauthoryear{Hendershott and Slemrod}{1983}]{hendershott83}
\textup{Hendershott, P.H. and Slemrod, J.} % (1983):
``Taxes and the User Cost of Capital for Owner-Occupied Housing,''
\textit{Real Estate Economics}, 10:4 (1983), 375--393.
\endbibitem

\bibitem[\protect\citeauthoryear{Holcombe and Sobel}{2001}]{holcombe01}
\textup{Holcombe, R.G. and Sobel, R.S.} % (2001):
``Public Policy Toward Pecuniary Externalities,''
\textit{Public Finance Review}, 29:4 (2001), 304--325.
\endbibitem

\bibitem[\protect\citeauthoryear{Kanemoto}{1988}]{kanemoto88}
\textup{Kanemoto, Y.} % (1988):
``Hedonic Prices and the Benefits of Public Projects,''
\textit{Econometrica}, 56:4 (1988), 981--989.
\endbibitem

\bibitem[\protect\citeauthoryear{Kuminoff and Pope}{2014}]{kuminoffpope14}
\textup{Kuminoff, N.V., and Pope, J.C.} % (2014):
``Do Capitalization Effects Measure the Willingness to Pay for Public Goods?,''
\textit{International Economic Review}, 55:4 (2014), 1227--1250.
\endbibitem

\bibitem[\protect\citeauthoryear{Levinson}{2021}]{levinson21}
\textup{Levinson, A.} % (2021):
``America's Regressive Wealth Tax: State and Local Property Taxes,''
\textit{Applied Economics Letters}, 28:14 (2021), 1234--1238.
\endbibitem

%\bibitem[\protect\citeauthoryear{Livy}{2018}]{livy18}
%\textup{Livy, M.R.} (2018):
%``Intra-School District Capitalization of Property Tax Rates,''
%\textit{Journal of Housing Economics}, 41, 227--236.
%\endbibitem

\bibitem[\protect\citeauthoryear{McMillen and Singh}{2020}]{mcmillen20}
\textup{McMillen, D. and Singh, R.} % (2020):
``Assessment Regressivity and Property Taxation,''
\textit{Journal of Real Estate Finance and Economics}, 60 (2020), 155--169.
\endbibitem

\bibitem[\protect\citeauthoryear{Mohai, Pellow and Timmons Roberts}{2009}]{mohai09}
\textup{Mohai, P., Pellow, D., and Timmons Roberts, J.} % (2009):
``Environmental Justice,''
\textit{Annual Review of Environment and Resources}, 34 (2009), 405--430.
\endbibitem

\bibitem[\protect\citeauthoryear{Niskanen and Hanke}{1977}]{niskanen77}
\textup{Niskanen, W.A. and Hanke, S.H.} % (1977):
``Land Prices Substantially Underestimate the Value of Environmental Quality,''
\textit{The Review of Economics and Statistics}, 59:3 (1977), 375--377.
\endbibitem

\bibitem[\protect\citeauthoryear{Palmquist}{1984}]{palmquist84}
\textup{Palmquist, R.B.} % (1984):
``Estimating the Demand for the Characteristics of Housing,''
\textit{The Review of Economics and Statistics}, 66:3 (1984), 394--404.
\endbibitem


%\bibitem[\protect\citeauthoryear{Palmquist}{1989}]{palmquist89}
%\textup{Palmquist, R.B.} (1989):
%``Land as a Differentiated Factor of Production: A Hedonic Model and its Implications for Welfare Measurement,''
%\textit{Land Economics}, 65(1), 23--28.
%\endbibitem

\bibitem[\protect\citeauthoryear{Poterba}{1984}]{poterba84}
\textup{Poterba, J. M.} % (1984):
``Tax Subsidies to Owner-Occupied Housing: An Asset-Market Approach,''
\textit{The Quarterly Journal of Economics}, 99:4 (1984), 729--752.
\endbibitem

\bibitem[\protect\citeauthoryear{Rosen}{1974}]{rosen74}
\textup{Rosen, S.} % (1974):
``Hedonic Prices and Implicit Markets: Product Differentiation in Pure Competition,''
\textit{The Journal of Political Economy}, 82:1 (1974), 34--55.
\endbibitem

%\bibitem[\protect\citeauthoryear{Samuelson}{1939}]{samuelson39}
%\textup{Samuelson, P.} (1939):
%``Interactions between the Multiplier Analysis and the Principle of Acceleration,''
%\textit{The Review of Economics and Statistics}, 62(2), 75--78.
%\endbibitem

\bibitem[\protect\citeauthoryear{Scitovsky}{1954}]{scitovsky54}
\textup{Scitovsky, T.} % (1954):
``Two Concepts of External Economies,''
\textit{Journal of political Economy}, 21:2 (1954), 143--151.
\endbibitem


\bibitem[\protect\citeauthoryear{Sirmans, Gazlaff, and Macpherson}{2008}]{sirmans08}
\textup{Sirmans, G.S., Gatzlaff, D.H., and Macpherson, D.A.} % (2008):
``The History of Property Tax Capitalization in Real Estate,''
\textit{Journal of Real Estate Literature}, 16:3 (2008), 327--343.
\endbibitem


\bibitem[\protect\citeauthoryear{Smith and Huang}{1995}]{smithhuang95}
\textup{Smith, V.K. and Huang, J.C.} % (1995):
``Can Markets Value Air Quality? A Meta-Analysis of Hedonic Property Value Models,''
\textit{Journal of Political Economy}, 103:1 (1995), 209--227.
\endbibitem


\bibitem[\protect\citeauthoryear{Sonstelie and Portney}{1980}]{sonstelie80}
\textup{Sonstelie, J. C., and Portney, P. R.} % (1980):
``Gross Rents and Market Values: Testing the Implications of Tiebout's Hypothesis,''
\textit{Journal of Urban Economics}, 7:1 (1980), 102-118.
\endbibitem

\bibitem[\protect\citeauthoryear{Tiebout}{1956}]{tiebout56}
\textup{Tiebout, C.M.} % (1956):
``A Pure Theory of Local Expeditures,''
\textit{Journal of Political Economy}, 64:5 (1956), 416-424.
\endbibitem



\bibitem[\protect\citeauthoryear{Wolch, Byrne, and Newell}{2014}]{wolch14}
\textup{Wolch, J. R., Byrne, J., and Newell, J. P.} % (2014):
``Urban Green Space, Public Health, and Environmental Justice: The Challenge of Making Cities ‘Just Green Enough’,''
\textit{Landscape and Urban Planning}, 125 (2014), 234--244.
\endbibitem



\bibitem[\protect\citeauthoryear{Zabel and Kiel}{2000}]{zabelkiel00}
\textup{Zabel, J.E., Kiel, K.E.} % (2000):
``Estimating the Demand for Air Quality in Four U.S. Cities,''
\textit{Land Economics}, 76:2 (2000), 174--194.
\endbibitem

\bibitem[\protect\citeauthoryear{Zilberman, et al.}{2013}]{zilberman13}
\textup{Zilberman, D., Barrows, G., Hochman, G. and Rajagopal, D.} % (2013):
``On the Indirect Effect of Biofuel,''
\textit{American Journal of Agricultural Economics}, 95:5 (2013), 1332--1337.
\endbibitem

\bibitem[\protect\citeauthoryear{Zilberman, Hochman, and Rajagopal}{2011}]{zilberman11}
\textup{Zilberman, D., Hochman, G. and Rajagopal, D.} % (2011):
``On the Inclusion of Indirect Land Use in Biofuel,''
\textit{University of Illinois Legal Review}, 2011:2 (2011), 413--434.
\endbibitem
47204720


\bibitem[\protect\citeauthoryear{Zodrow}{2001}]{zodrow01}
\textup{Zodrow, G. R.} % (2001):
``The Property Tax as a Capital Tax: A Room with Three Views,''
\textit{National Tax Journal}, 54:1 (2001), 139--156.
\endbibitem


\bibitem[\protect\citeauthoryear{Zodrow}{2014}]{zodrow14}
\textup{Zodrow, G. R.} % (2014):
``Intrajurisdictional Capitalization and the Incidence of the Property Tax,''
\textit{Regional Science and Urban Economics}, 45 (2014), 57--66.
\endbibitem


%example references
%
%\bibitem[\protect\citeauthoryear{Aumann}{1987}]{b1}
%\textsc{Aumann, R. J.} (1987):
%``Correlated Equilibrium as an Expression of Bayesian Rationality,''
%\textit{Econometrica}, 55, 1--18.
%\endbibitem
%
%\bibitem[\protect\citeauthoryear{Peck}{1994}]{b2}
%\textsc{Peck, J.} (1994):
%``Competition in Transactions Mechanisms: The Emergence of Competition,''
%Unpublished Manuscript, Ohio State University.
%\endbibitem
%
%\bibitem[\protect\citeauthoryear{Enelow and Hinich}{1990}]{b3}
%\textsc{Enelow, J., and M. Hinich}, eds. (1990):
%\textit{Advances in the Spatial Theory of Voting}.
%Cambridge, U.K.: Cambridge University Press.
%\endbibitem
%
%\bibitem[\protect\citeauthoryear{Wittman}{1990}]{b4}
%\textsc{Wittman, D.} (1990):
%``Spatial Strategies when Candidates Have Policy Preferences,''
%in \textit{Advances in the Spatial Theory of Voting},
%ed. by M. Hinich and J. Enelow.
%Cambridge, U.K.: Cambridge University Press, 66--98.
%\endbibitem
%
%\bibitem[\protect\citeauthoryear{Cahuc, Postel-Vinay and Robin}{2006}]{b5}
%\textsc{Cahuc, P., F. Postel-Vinay, and J.-M. Robin} (2006): 
%``Supplement to `Wage Bargaining with On-the-Job Search: Theory and Evidence',''
%\textit{Econometrica Supplementary Material}, 74.
%\endbibitem
\end{thebibliography}

\end{document}
